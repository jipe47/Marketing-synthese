
\chapter{Les nouveaux produits}
	
	\section{Les étapes du processus de développement}

	\dessinS{22}{0.8}
	\dessinS{53}{0.65}
	
	Il y a 4 grandes étapes :
	
	\begin{enumerate}
		\item La recherche d'idées
		\item Stade du concept
		\item Stade du prototype
		\item Stade de lancement
	\end{enumerate}
	
		\subsection{Recherche d'idées}
		
		Tueurs d'idée : on a déjà essayé, ça prendrait top de temps, trop cher, ce n'est pas mon/votre travail, ...
		
		\dessinS{54}{0.65}
		
		Attention aux erreurs d'évaluation.
		
		
		Techniques de créativité :
		
		\begin{enumerate}
			\item brainstorming
			\item analyse fonctionnelle (morphologie)
			\item le client (lead-user)
			\item analogie
			\item segmentation par avantages recherchés
			\item réalité virtuelle
			\item carnet d'idées
			\item etc
		\end{enumerate}
	
		
			\subsubsection{Brainstorming}
			
			\begin{enumerate}
				\item générer des idées
				
				\begin{itemize}
					\item maintenir une ambiance libre et détendue
					\item encourager la participation de tous
					\item se concentrer sur la quantité uniquement
					\item valoriser l'originalité et l'exubérance (idée la plus folle ? L'idée qui me mettrait à la porte ?)
					\item interdire la critique
				\end{itemize}
				
				\item trier les idées
				
				\begin{itemize}
					\item faire connaître les critères d'évaluation (finance, personnel, \dots )
					\item ramener es idées extravagantes à des dimensions pratiques
					\item juger les idées après légères modifications
				\end{itemize}
				
				\item assurer le suivi
				
				\begin{itemize}
					\item remercier les membres du groupe pour leur participation
					\item valoriser tout le groupe si une solution émise est retenue
					\item répondre individuellement pour les solutions non retenues (positif ("votre idée est bonne"), possibilités ("elle pourrait"), réserves ("mais.. comment le surmonter"))
				\end{itemize}
			\end{enumerate}
		
		
			\subsubsection{L'analyse morphologique}
			
			Identifier les dimensions les plus importantes d'un produit et ensuite examiner deux à deux les relations entre chaque dimension.
			
			\subsubsection{Le client}
			
			Utiliser le client B2B comme source d'idées.
			
			\dessin{23}
			
			
			\subsubsection{L'analogie}
			
			\begin{enumerate}
				\item poser le problème
				\item créer des analogies
				\item résoudre l'analogie
				\item transférer les solutions au problème
			\end{enumerate}
			
			
			
		\subsection{Le filtrage des idées}
		
		Éliminer rapidement et tôt les idées de produits incompatibles avec les ressources ou les objectifs de l'entreprise ou peu attractives. Généralement, on utilise une grille d'évaluation, avec des critères propres à chaque département, un niveau minimum requis et des pondérations.
		
\begin{comment}
		\begin{enumerate}
			\item grille d'évaluation (critères propres à chaque département ; niveau minimal requis ; pondérations)
			\item évaluations des idées ...
			\item ...
			\item ...
		\end{enumerate}
\end{comment}

		\subsection{Le concept}
				
		Le concept de produit est une description de préférence visuelle et verbale des caractéristiques physiques et perceptuelles du produit final envisagé et de la promesse qu'il est censé tenir, pour un groupe particulier d'acheteurs.
			
			
		\subsection{Annexes}
			\subsubsection{L'analyse conjointe}
					
			\begin{itemize}
				\item[+] situation réaliste, proche du comportement (jugement global, présentation concrète et compromis)
				\item[+] mesure indirecte du système des valeurs
				\item[+] limitation des effets de rationalisation
				\item[-] capacité cognitive du répondant limitée
				\item[-] enquête lourde (fatigue, inconsistance, abandon)
				\item[-] procédure délicate (construction, estimation)
			\end{itemize}
			
			$\Longrightarrow$ outil limité aux améliorations de produits
								
	\section{Les taux de succès des nouveaux produits}
	
		
	Exemple d'étude permettant de dire si un produit aura du succès ou non :
	
	\begin{enumerate}
		\item produit supérieur
		\item fort orientation marché
		\item concept global
		\item analyse préliminaire
		\item définition précise du concept
		\item plan de lancement structuré
		\item coordination inter-fonctionnelle
		\item soutien direction générale
		\item utilisation des synergies
		\item attractivité des marchés
		\item présélection des projets
		\item qualité suivi du lancement
		\item disponibilité des ressources
		\item rapidité
		\item procédures multi-échelons
	\end{enumerate}
		

	\section{Processus de diffusion des innovations}
	
	Les attributs perçus du produit facilitant la réceptivité à l'innovation :
	
	\begin{enumerate}
		\item l'avantage relatif
		\item la compatibilité
		\item la faible complexité
		\item la possibilité d'essai
		\item l'observabilité
	\end{enumerate}
	
	\dessin{16}
	
	On peut distinguer 5 catégories d'individu :
	
	\begin{itemize}
		\item Innovateurs : adoption rapide même s'il y a un risque, peu influencés par les autres individus, esprit d'aventure et peu sensibles au prix ;
		\item Adopteurs précoces : leader d'opinion, adoptent les idées et produits nouveaux avec prudence ; 
		\item Majorité précoce : besoin d'information sur l'innovation, avec lenteur dans le processus d'analyse ;
			
		\item Majorité tardive : scepticisme, moins cosmopolites et réactifs au changement, prudents, poussés par les autres individus ;
			
		\item Retardataires : tournés vers le passé, craignent toute évolution, attentifs au prix.
	\end{itemize}
	
	\begin{comment}

		Caractéristique des premières catégories d'adopteurs :
		\begin{itemize}
			\item status socioéconomique : plus longue éducation, .................
			\item personnalité
			\item communication
		\end{itemize}
	\end{comment}
		
		
	\section{La prévision de la demande d'un nouveau produit}
		
	\dessinS{24}{0.65}
		
	Nouveaux adopteurs = innovateurs + imitateurs
		
	$$a_t = p(N - A_{t - 1}) + q( \frac{A_{t - 1}}{N} ) ( N - A_{t - 1})$$

	Avec :
		
	\begin{itemize}
		\item $a_t$ : nbr de nouveaux acheteurs en $t$ ;
		\item $A_t$ : nbr acheteur ayant adopté en t - 1 ;
		\item $N$ : total acheteurs potentiels ;
		\item $N - A_{t - 1}$ : potentiel restant à capturer ;
		\item $ \frac{A_{t - 1}}{N}$ : pénétration déjà réalisée ;
		\item $p$ : taux d'innovation = part des acheteurs encore à capturer qui adopteront par innovation ;
		\item $q$ : taux d'imitation = part des acheteurs encore à capturer qui adopteront par imitation.
	\end{itemize}
		
	Conclusions sur l'innovation et l'imitation :
	
	\begin{itemize}
		\item les taux d'innovation sont plus élevés pour les biens durables, les taux d'imitation pour les biens industriels ;
		\item dans les pays à culture collectiviste, plutôt qu'individualiste (Japon plutôt que USA), le taux d'imitation est plus élevé ;
		\item dans les pays à haut pouvoir d'achat par habitant, le taux d'innovation est plus élevé ;
		\item les produits à effet de réseau (ex : fax, internet) ont un taux d'imitation plus élevé. Parallèlement, taux d'innovation tend aussi à être plus lent, rendant la source en S plus prononcée.
	\end{itemize}
		
	% Qu'était-il marqué sur le transparent ? Pourquoi tant de haine ? FFFFFFUUUUUUU ? La phrase qui suit restera inachevée...
	% Les deux taux....
		
				
	\section{Les mesures du succès des produits nouveaux}
	
	Période de remboursement : quand vais-je récupérer l'investissement (années) ?
	
	$$\text{payback} = \frac{\text{investissement}}{\text{profits annuels}}$$
	
	Taux interne de rentabilité : que me rapporte l'investissement (\% annuel) ?
	
	$$\text{rentabilité} = \frac{\text{profits annuels}}{\text{investissements}}$$
		
	$$\text{Indice de R\&D} = \frac{  \text{\% chiffre d'affaire généré par les NP} . (\text{ \% marge} + \text{\% R\&D})}{ \text{pourcentage R\&D}}$$
		
	Si indice R\&D $\geq$ 1, ok, sinon perte.
		