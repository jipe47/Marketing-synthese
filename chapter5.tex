
\chapter{L'analyse de la compétitivité}

	\section{Notion d'avantage concurrentiel}
	
	\paragraph{Avantage concurrentiel externe}
	Avantage concurrentiel-qualité (ACE) : s'appuie sur des qualités distinctives du produit qui donnent une valeur au client, soit en réduisant ses coûts, soit en augmentant sa satisfaction ou sa performance (externe). Dilemme adaptation (coûtera plus cher mais vend mieux).\\
	
	"Dans quelle mesure nos clients sont-ils prêts à payer pour nos produits un prix de vente supérieur à celui du concurrent le plus dangereux ?" : logique de marketing dominant.
	
	\paragraph{Avantage concurrentiel interne}
	Avantage concurrentiel-coût (ACI) : s'appuie sur la supériorité de l'entreprise en matière de productivité qui donne une valeur au producteur en lui donnant un prix de revient inférieur à celui du concurrent prioritaire (interne). Ex : délocaliser, baisser la qualité. Dilemme standardisation (coûte moins cher, mais vend moins). \\
	
	"Comment se compare le prix de revient de nos produits à celui du concurrent le plus dangereux ?" : logique de production dominante.	
	
	%Logique de production dominante : comment se compare le prix de revient de nos produits à celui du concurrent le plus dangereux ? \\
		
	%Logique de marketing dominante : ?? \\
	
	

	\dessin{19}		
		
	Il vaut mieux être en dessous de la diagonale plutôt qu'au-dessus.
		
		%Types d'avantages concurrentiels détenus par les "champions discrets" : p271
		
	\section{Notion de rivalité élargie}
		

		
	%Courtier est plus général et défend mieux, tandis qu'un agent conseillera plus son agence.
	
	Idée : la capacité d'une entreprise à exploiter un avantage concurrentiel dans son marché dépend de la concurrence directe mais aussi de forces rivales telles que les entrants potentiels, les produits de substitution, les clients et les fournisseurs.
	
	\dessin{20}
	
		\subsection{La menace des nouveaux entrants}
		
		Entrants potentiels :
		
		\begin{itemize}
			\item les firmes pour lesquelles une entrée constituerait une synergie manifeste ;
			\item les firmes dont l'entrée est le prolongement logique de leur stratégie ;
			\item les clients ou fournisseurs qui peuvent s'intégrer en amont ou en aval.
		\end{itemize}
		\n
		Pour se protéger, on fixe des barrières à l'entrée :
		
		\begin{itemize}
			\item les économies d'échelle : contrainte de démarrer sur une vaste échelle ;
			\item les brevets ;
			\item la différenciation du produit et la force du capital de marque : entrainent une fidélité élevée ;
			\item les besoins en capitaux ;
			\item le coût de transfert, l'accès aux circuits de distribution, l'effet d'expérience, etc.
		\end{itemize}
		\n
		La force de la riposte des concurrents établis sur un marché envers un entrant potentiel dépendra 
		
		\begin{itemize}
			\item du passé et de la réputation d'agressivité vis-à-vis des nouveaux entrants ;
			\item le degré d'engagement de la firme établie dans le produit-marché ;
			\item la disponibilité des ressources financières ;
			\item une capacité de représailles sur le marché du même entrant.
		\end{itemize}
	
		 \subsection{La menace des produits de substitution}
		 
		 Menace permanente, surtout quand le rapport qualité/prix du produit de substitution est meilleur. Il faut pratiquer une veille technologique pour surveiller le marché.
		 
		 \subsection{Le pouvoir de négociation des clients}
		 
		 Les clients peuvent influencer la rentabilité d'une entreprise en contraignant l'entreprise à baisser les prix, en exigeant des services plus étendus, de meilleurs conditions de paiement ou en dressant les concurrents les uns contre les autres.
		 
		 L'importance de ce pouvoir de négociation dépend de certaines conditions :
		 
		 \begin{itemize}
		 	\item groupe de clients concentré, ou achète des quantités importantes par rapport au chiffre d'affaires (ex : grande distribution) ;
		 	\item les produits achetés représentent une part très importante de son propre coût, ce qui le fera négocier durement ;
		 	\item les produits achetés sont peu différenciés, et les clients sont sûrs de pouvoir trouver d'autres fournisseurs ;
		 	\item les coûts de transfert ($\Leftrightarrow$ coûts de changement de fournisseur) sont faibles pour le client ;
		 	\item les clients sont une réelle menace d'intégration vers l'amont, et sont des entrants potentiels dangereux ;
		 	\item les clients disposent d'une information complète sur la demande, les prix réels du marché et même les coûts du fournisseur.
		 \end{itemize}
		 
		 Une firme peut pratiquer une politique de sélection de sa clientèle, pour éviter d'être dépendant du groupe de clients.
		 
		 \subsection{Le pouvoir de négociation des fournisseurs}
		 
		 Les fournisseurs ont la possibilité d'augmenter les prix, de réduire la qualité ou la quantité du matériel vendu. Conditions qui assurent un pouvoir de négociation élevé à un fournisseur (symétriques à celles prévalant le pouvoir des clients) :
		 
		 \begin{itemize}
		 	\item le groupe de fournisseurs est plus concentré que le groupe de clients auquel il vend ;
		 	\item le fournisseur n'est pas menacé par des produits susceptibles de se substituer aux produits qu'il vend ;
		 	\item l'entreprise n'est pas un client important du fournisseur ;
		 	\item le produit est un moyen de production important pour le client ;
		 	\item le groupe de fournisseurs a différencié ses produits ou a créé des coûts de transfert, qui rendent le client captif ;
		 	\item le groupe de fournisseurs constitue une réelle menace d'intégration vers l'aval.
		 \end{itemize}
		 
	\section{Les situations concurrentielles}
	
	\dessinS{59}{.6}
		
	\begin{enumerate}
		\item Concurrence parfaite : beaucoup de vendeurs et d'acheteurs. Les produits indifférenciés sont parfaitement substituables, et il y a une absence complète de pouvoir de marché.
		\item Oligopole : dépendance entre firmes rivales très fortes, à cause du faible nombre de concurrents ou de la présence de quelques entreprises dominantes. Les actions d'un concurrent se font très vite ressentir. On distingue deux types d'oligopole :
		
		\begin{itemize}
			\item oligopole indifférencié : la dépendance entre concurrents est d'autant plus forte que les produits des firmes sont indifférenciés ;
			\item oligopole différencié : les biens ont des qualités distinctives importantes pour l'acheteur.
		\end{itemize}
		
		\item Concurrence monopoliste : concurrents nombreux et force équilibrée, mais les produits sont différenciés (caractéristiques distinctives importantes pour l'acheteur). Stratégie de différenciation basée sur un avantage concurrenciel externe. 
		
		\item Monopole : un seul producteur, grand nombre d'acheteur. Il y a deux types de monopole : 
			\begin{itemize}
				\item monopole de l'innovateur : monopole en phase d'introduction du cycle de vie d'un produit
				\item monopole d'Etat : la logique de profit fait la place à celle de l'intérêt général. Objectifs difficiles à faire respecter, car pas de verdict du marché : cela favorise les préoccupations internes, au détriment des besoins des usagers.
			\end{itemize}
	\end{enumerate}
	
	
	\section{L'avantage concurrentiel externe par la différenciation}
	
	Tout produit est un panier d'attributs (un panier de services) et des occasions de différenciation existent toujours. \\
		
	Un marché n'est jamais tout à fait homogène, au moins 3 types de segments existent toujours :
		
	\begin{itemize}
		\item les clients sensibles uniquement au prix (irréductibles)
		\item les clients sensibles aux services ou à l'assistance (occasionnels)
		\item les clients focalisés sur des critères spécifiques (exigeants)
	\end{itemize}
		
		
		\subsection{La différenciation}
		
		Une entreprise se différencie de ses concurrents lorsqu'elle offre quelque chose d'unique (autre qu'un prix faible) représentant une valeur pour les acheteurs, lesquels sont prêts à payer un prix supérieur à celui de la concurrence.
		
		Les sources de différenciation sont multiples, et peuvent provenir de n'importe quelle activité de l'entreprise. \\
		
		
		Conditions de réussite de la différenciation :
		
		\begin{enumerate}
			\item la valeur doit être apportée à l'acheteur soit en augmentant la performance du produit soit par une diminution de ses coûts ;
			\item l'unicité doit porter sur une caractéristique importante pour le client ;
			\item le supplément de prix doit trouver une justification aux yeux du client ;
			\item le supplément de prix doit être supérieur au supplément de coût causé;
			\item la différenciation doit être défendable, c'est-à-dire à l'abri des imitations ;
			\item il faut connaître les éléments d'unicité cachés par des signaux.
		\end{enumerate}
		
		%[exemple p267]
		
		\subsection{Comment mesurer son pouvoir de marché ?}
		\begin{itemize}
			\item une plus faible sensibilité au prix
			\item un différentiel de prix acceptable
			\item un taux d'exclusivité supérieur
			\item un taux de fidélité supérieur
			\item des mesures d'attitudes positives (familiarité, estime, préférence, intention)
		\end{itemize}	
		
	\section{L'avantage interne et l'effet d'expérience}
	
	Sources de réductions de coût :
	
	\begin{itemize}
		\item économie d'échelle : indivisibilités, spécialisation et division du travail
		\item effet d'expérience : dextérité accrue, amélioration des coordination et de l'origine, proximité d'une source d'inputs
		\item coût des inputs : propriété d'une source à faible coût, pouvoir de négociation, arrangement facilitait coordination et faibles coûts de transfert (?)
		\item utilisation de la capacité : ratio coûts fixes/variables, coûts d'ouverture/fermeture de capacité
		\item technique de production : mécanisation et automatisation, utilisation plus efficiente des matières premières, précision améliorées (moins de déchets)
		\item meilleur conception et de design du produit
		\item efficience organisationnelle.
	\end{itemize}
	
		\subsection{Effet d'expérience}
	
		% (p272)
		Le coût unitaire de la valeur ajoutée d'un produit homogène, mesurée en unités monétaires constantes, diminue avec l'augmentation de l'expérience, mesurée par la production cumulée. 
		
		Cette diminution correspond à un pourcentage fixe (donc prévisible) à chaque fois que la production cumulée double. \\
		
		L'effet d'expérience n'est pas un effet d'échelle : l'effet d'expérience se manifeste avec le temps, tandis que l'effet d'échelle se manifeste avec la taille d'une activité. De plus, les avantages d'expérience ne se manifestent pas spontanément (des efforts sont nécessaires), alors que l'effet d'échelle divise les coûts fixes "automatiquement".
		%Exemple p 274.
		
			\subsubsection{Conditions d'application}
		
		\begin{enumerate}
			\item loi volontariste... : les coûts ne baissent que si on veut les faire baisser
			\item ... liée à la capacité humaine de s'améliorer (VA) : l'effet porte surtout sur les coûts de la valeur ajoutée ;
			\item ... pour une production homogène : si changement radical de produit, l'effet est perdu.
		\end{enumerate}
		.. qui peut s'ajouter à l'effet d'échelle. \\
		
		Il devient moins visible :
		
		\begin{itemize}
			\item lorsque la production n'en est plus à son début
			\item l'inflation est importante
			\item la concurrence le contourne par d'autres économies ou par une différenciation
		\end{itemize}
		
		
			\subsubsection{Interprétation}
		
		\begin{itemize}
			\item "coût unitaire" : le coût de la dernière unité
			\item coûts hors inflation
			\item entre deux doublement, il y a aussi réduction, mais elle est moins facile à calculer
			\item expérience = production cumulée
			\item production cumulée $\neq$ production par période
			\item porte sur la valeur ajoutée
		\end{itemize}
		
			\subsubsection{Principales causes de l'effet d'expérience}
		
		\begin{itemize}
			\item efficacité du travail manuel ;
			\item spécialisation du travail et des méthodes ;
			\item nouveaux procédés de fabrication ;
			\item meilleur équipement de production ;
			\item modification des ressources utilisées ;
			\item nouvelle conception du produit.
		\end{itemize}
		
		\n
		Soient :
		\begin{itemize}
			\item $C_p$ le coût unitaire prévu (p)
			\item $C_b$ le coût unitaire de base (b)
			\item $Q_p$ la quantité cumulée prévue
			\item $Q_b$ la quantité cumulée de base
			\item $ \epsilon$ l'élasticité du coût
		\end{itemize}		
		
		$$C_p = C_b \times (\frac{Q_p}{Q_b})^{-\epsilon}$$
		
		
	%	Fixation du prix en fonction de la courbe d'expérience (p277, 466) : pas bien !!
		
		%-> Prix de pénétration et d'écrémage 
		
			\subsubsection{Fixation du prix en fonction de la courbe d'expérience}
			
			On pratique 3 types de fixation de prix.
			
			\dessin{17}
			
			
			\paragraph{Le prix de pénétration}
			Anticipation de l'évolution du coût unitaire, l'objectif est une croissance des ventes plus rapide que celle du marché. Une fois le niveau d'expérience atteint, la baisse des coûts se répercute sur le prix ;
			
			\begin{itemize}
				\item[+] permet de gagner en parts de marché relatives
				\item[+] donc de s'accorder sur un avantage concurrentiel coût (ACI)
				\item[-] risque qu'il n'existe pas d'effet d'expérience
				\item[-] impossibilité de remonter les prix en cas d'erreur
			\end{itemize}
			
			\paragraph{Le prix d'écrémage}
			Privilégier les profits à court terme au détriment d'un avantage concurrentiel interne ;
			
			\begin{itemize}
				\item[+] convient aux pionniers et adopteurs précoces(faible sensibilité au prix)
				\item[+] permet d'écrémer et répond au besoin de liquidités en introduction 
				\item[+] si erreur d'évaluation, moindre risque financier et possibilité de corriger 
				\item[-] risque de ne pas repérer à temps la nécessité de baisser
				\item[-] rend le marché plus attractif pour la concurrence
			\end{itemize}
		
			\paragraph{Prix proportionnel au coût}
		
		
		
			\subsubsection{Limite de la loi d'expérience}
		
		Situations où la loi d'expérience ne se manifeste que très peu :
		\begin{itemize}
			\item le potentiel d'apprentissage est peu élevé ou la part de valeur ajoutée dans le produit est faible ;
			\item l'impact de l'effet d'expérience neutralisé par un concurrent avec faible part de marché mais bénéficiant d'une réduction de coût plus importante ;
			\item un concurrent avec une faible part de marché et bénéficiant d'un effet d'expérience fort grâce à une supériorité technologique ;
			\item différences d'expérience annihilées par des innovations dans le produit ou le procédé ;
			\item l'effet d'expérience ne peut être exploité à cause d'une faible sensibilité au prix du marché ;
			\item un concurrent bénéficie d'une source d'approvisionnement privilégiée, avantage-coût ;
			\item facteurs stratégiques autres que la part du marché, par exemple des interventions légales.
		\end{itemize}
		
		
			\subsubsection{Tirer parti de l'effet d'expérience}
			
		%Avantages et inconvénients : [INSERT PAGE NUMBER (dans les 400 ?)]
		
		%Comment tirer parti :
		%	- indicateur à l'avance
		%	- avantage concurrent-[...]
		%		* ?
		%		* ?
		%		* ?
		%		* ?
			\dessinS{60}{.7}
		
			\subsubsection{Se protéger de l'effet d'expérience}
		\begin{itemize}
			\item différenciation
			\item autre avantage interne
			\item faible valeur ajouté
			\item interventions légales, protectionnisme
			\item technologie nouvelle
		\end{itemize}
		