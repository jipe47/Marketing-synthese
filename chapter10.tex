
\chapter{Le prix}
	
	\section{Rôle du prix dans la stratégie marketing}
	
	\dessinS{36}{0.68}
		
		\subsection{Le prix vu sous l'angle du client}
		
		Expression monétaire de la valeur, occupe une position centrale dans les échanges concurrentiels. 
		
		$$\text{prix} = \frac{\text{ensemble des sacrifices monétaires et non monétaires}}{\text{ensemble des satisfactions reçues}}$$
		
		
		On peut modifier
		
		\begin{itemize}
			\item la quantité d'argent demandée par le vendeur ;
			\item la quantité de biens ou de services offerte par le vendeur ;
			\item la qualité des biens et services fournis par le vendeur
			\item les primes et réductions pour quantités achetées
			\item le moment et le lieu du transfert du titre de propriété
			\item le lieu et le moment du paiement
			\item les modes de paiement (carte de crédit, chèques)
			\item les conditions de paiement (comptant, 30 jours, etc)
		\end{itemize}
		\n
		Prix nominal : coût monétaire
		
		terme de l'échange : modalités, délais de paiement, délais de livraison, service après-vente, coût de négociation
		
		coût de transfert monétaire : coûts d'adoption, de formation
		
		coût de ...
		
		\subsection{Notion de coût de transfert}
		
		\begin{itemize}
			\item coût de modification des produits existants pour correspondre au produit du nouveau fournisseur ;
			\item changement dans les habitudes ;
			\item dépenses de formation, temps de formation
			\item investissement dans les nouveaux équipements pour utiliser les nouveaux produits
			\item coûts de réorganisation
			\item coûts psychologiques du changement
		\end{itemize}
		
		\subsection{Le rôle du prix dans la stratégie}
		
		L'importante des décisions prix
		
		\begin{itemize}
			\item le prix détermine le niveau de la demande et donc de l'activité
			\item le prix détermine la rentabilité directe (la marge) et indirecte (couverture des frais fixes)
			\item prix perçu comme un signal de positionnement de la marque
			\item le prix se prête facilement à des comparaisons entre produits concurrents
			\item le prix choisi doit être compatible avec les autres composants de la stratégie marketing
		\end{itemize}
		\n
		Les objectifs des stratégies de prix :
		
		\begin{itemize}
			\item les objectifs centrés sur l'entreprise : maximisation du profit, profit suffisant, contrainte de profit minimum
			\item les objectifs centrés sur la concurrence : alignement (prix de parité), mise hors marché (prix prédateur), concurrence hors-prix
			\item les objectifs centrés sur la demande : prix à la valeur perçue, prix optimal, prix flexible
		\end{itemize}
		
		
	\section{Annexe : notion d'élasticité}
		
		$$\text{élasticité} = \frac{\text{pourcentage de réaction (conséquence)}}{\text{pourcentage d'action (cause)}} $$
		
		Génération négative et $<$ -1.
		
		\subsection{Type d'élasticité}
		
		\begin{itemize}
			\item Elasticité de part de marché :
			
			$$\nu q, p = \frac{ \text{\% variation q} }{\text{\% variation p}} = \frac{ \frac{\Delta q}{q} }{ \frac{\Delta p}{p} }$$
			
			\item Elasticité croisée : impact d'une variation du prix du concurrent sur la quantité du produit
				$$\nu q_i, pj =  \frac{ \frac{\Delta q_i}{q_i} }{ \frac{\Delta p_j}{p_j} }$$
				
				avec $q_i$ la quantité de notre produit et $p_j$ le prix du concurrent.
				
			\item Elasticité de réaction : impact d'une variation du prix du concurrent  sur le prix du produit.
			
			$$\nu p_j, pi =  \frac{ \frac{\Delta p_j}{p_j} }{ \frac{\Delta p_i}{p_i} }$$
				
				avec $p_j$ le prix du concurrent et $p_i$ notre prix.
				
				
				
			\item Elasticité de court et de long terme
			
			$$Q = \underbrace{\underbrace{P_t^\alpha}_{\text{CT}} \underbrace{P_{t - 1}^\beta P_{t - 2}^\gamma P_{t - 3}^\delta P_{t - 4}^\epsilon}_{\text{différé}}}_{\text{long terme}}$$
			
			avec $Q$ la quantité et $P$ le prix.
			
			\item Elasticité critique : élasticité nécessaire à la réalisation d'un objectif ou au respect d'une contrainte.
		\end{itemize}
		
		\dessinS{38}{0.65}	
		
		
		\subsection{Sensibilité du prix}
		
		
		10 déterminants de la sensibilité au prix
		
		\begin{itemize}
			\item effet de valeur unique (- : réduit la sensibilité) ;
			\item effet de notoriété des substituts (+) ;
			\item effet de comparaison difficile (-) ;
			\item effet de dépense totale (+)
			\item effet de l'avantage final (-)
			\item effet de coût partagé (-)
			\item effet d'investissement perdu (-)
			\item effet qualité-prix (-)
			\item effet de stock (+)
			\item sentiment d'injustice (+)
		\end{itemize}
		
		\subsection{Estimation de l'élasticité-prix}
		
		Méthodes d'estimation de l'élasticité-prix : 
		
		\begin{itemize}
			\item jugements d'experts (probabiliste ou déterministe) ;
			\item enquêtes consommateurs (directe ou analyse conjointe) ;
			\item expérimentation prix (terrain ou laboratoire) ;
			\item études économétriques (séries chronologiques ou panels consommateurs).
		\end{itemize}
		
		\dessinS{37}{0.65}
		
	\section{Prix sous l'angle des coûts}
	
	
	
		\subsection{Les prix internes}
		
		On calcule le prix sur la base des coûts, sans référence explicite aux données du marché
	
	
		\dessinS{39}{0.65}
		
		Soient $C$ le coût variable, $F$ les frais fixes, $Q$ l'hypothèse de volume, $K$ le capital investit et $r$ le taux de rentabilité suffisant
			\subsubsection{Prix plancher}
			
			
			Prix correspondant au coûts variables (ou directs)
	
			$$\text{Prix limite} = C$$
	
			\subsubsection{Prix technique}
			
			Prix correspondant au profit juste nul, c'est-à-dire prix qui recouvre tous les coûts (variables et fixes)
						
			$$\text{Prix technique} = C + \frac{F}{Q}$$
			
			\subsubsection{Le prix cible}
	
			Prix comprenant , outre les coûts variables et fixes, une contrainte de profit jugé suffisant.
	
	
	
			$$\text{Prix cible} = F + \frac{f + rK}{Q}$$
			$$\text{Prix cible (m)} = \frac{\text{Prix limite}}{1 - \text{taux de marge brute voulue}} = \frac{\text{Prix technique}}{1 - \text{taux de marge nette voulu}}$$
	
	
	Augmentation requise des ventes pour laisser la marge brute antérieure inchangée : $\frac{\delta Q}{Q} = \frac{ \frac{\delta P}{P}}{MB + \frac{\delta P}{P}}$
	
	avec $ \frac{\delta Q}{Q}$ = augmentation de quantité requise en $\%$
	$ \frac{\delta P}{P}$ = baisse de prix envisagée en $\%$
	MB = marge brute du prix avant la baisse en $\%$.
	
	Ex : $\frac{\delta P}{P}$ = 8$\%$, MB = $35 \%$, $\frac{\delta Q}{Q} = -18.6\%$
	
	
		\subsection{Prix des nouveaux produits}
		
		Prix de pénétration : prix bas pour avoir dès le départ une part de marché importante. \\
	
		Prix d'écrémage : prix élevé en se limitant aux groupes de clients prêts à payer le prix fort, pour assurer des rentrées financières importantes.
	
	\section{Prix sous l'angle de la demande}

	Dans l'orientation-marché, le prix est fixé dès le début
	
	\dessinS{40}{0.65}

	
		\subsection{Prix de positionnement (PP)}
		
		Basé sur la qualité, le prestige ; effet-qualité.
		
		
		\subsection{Prix psychologique optimum (PPO)}
		
		Ce prix s'applique lorsque deux effets co-existent :
		
		\begin{itemize}
			\item effet-qualité ou -prestige : "quel est le prix en dessous duquel le produit ne vous semblerait pas de bonne qualité ?", prix minimum ;
			\item effet-dépense : "quel est le prix à partir duquel vous n'achetiez plus le produit ?", prix maximum.
		\end{itemize}
		
		\subsection{Prix optimal de Cournot (POC)}
		
		Prix qui optimise le profit
		
		$$P_{\text{cournot}} = C \frac{\nu}{\nu + 1}$$
		
		avec C le coût direct unitaire, $\nu$ l'élasticité-prix et $ \frac{\nu}{\nu +1 } $ = coefficient de majoration de coût
		
		
		\subsection{Prix proportionnel à la valeur perçue}
		
		$$PVP = \overline{P} \times \frac{F_k}{\overline{F}_l}$$
		
		$\overline{P}$ = prix moyen du marché (ou du concurrent)
		
		
		\subsection{Prix à l'avantage économique (PAE)}
		
		Prix qui annule l'économie de coût réalisée grâce au produit
		
		
		\subsection{Prix flexibles}
		
		\begin{itemize}
			\item Prix flexibles selon les marchés (second-market discounting)
			\item flexibilité en fonction de la saisonnalité (periodic or random discounting)
			\item prix promotionnels
			\item remises et rabais
		\end{itemize}
		\n
		Conditions d'efficacité de la pratique des prix dynamiques selon le moment (yield management)
		
		\begin{itemize}
			\item sensibilité au prix hétérogène (willingness-to-pay)
			\item marché segmentable (selon la sensibilité du prix)
			\item peu de possibilité d'arbitrage
			\item coût de segmentation faible face aux revenus supplémentaires attendus
			\item pas de perception d'injustice
		\end{itemize}
		
		
	\section{Prix sous l'angle de la concurrence}
	
	
	Situation de concurrence parfaite :
	
	Conditions nécessaires :
	\begin{itemize}
		\item produit homogène
		\item information parfaite
		\item atomicité producteurs
		\item pas de barrière à l'entrée
	\end{itemize}
	
	
	\dessinS{42}{0.65}
	
	
	\begin{itemize}
		\item oligopole différencié : POC, PVP, PAE
		\item NP : pénétration/écrémage
		\item légal : état
		\item concurrence monopolistique : POC, PVP, PAE
		\item oligopole indifférencié : prix relatif au leader ou risque guerre de prix
		\item concurrence parfaite : prix du marché
	\end{itemize}
	
	\subsection{Mécanisme de la guerre des prix destructrice}
	\begin{comment}
	oligopole -> élasticité croisée élevée -> B perd des ventes, élasticité de réaction élevée -> B réduit ses prix
	marché à maturité -> élasticité de demande globale faible -> élasticité croisée élevée -> élasticité de réaction élevée
	
	[..] -> B perd des ventes -> B réduit ses prix
	\end{comment}
	
	\dessinS{56}{0.6}
	
	
	
	