
\chapter{Le ciblage et le positionnement}
	
	\section{Comment lire des cartes perceptuelles}
	
	
		%%% Annexe du chapitre 9]
		\subsection{L'analyse de similarités}
		
		On demande d'estimer des similarités entre des marques, en mettant des distances entre des marques.
		
		Il en vient une matrice.
		
		\begin{center}
		\begin{tabular}{|c|c|c|c|}
			\hline  & A & B & C \\ 
			\hline A & - &  &  \\ 
			\hline B & 2 & - &  \\ 
			\hline C & 9 & 6 & - \\ 
			\hline 
		\end{tabular} 
		\end{center}
		
		Nombre de dimensions nécessaires pour une reconstitution sans ambiguïté : nbrMarque - 1. On peut le diminuer, en perdant des données, pour pouvoir interpréter plus facilement.
		
		%[exemple dans l'annexe du chapitre 9]		
		
		Généralement :
		
		\begin{itemize}
			\item Axe X : 

			\begin{itemize}
				\item distribution réduite ou large ;
				\item image artisanale ou industrielle.
			\end{itemize}
			
			\item Axe Y :
			\begin{itemize}
				\item prix ;
				\item type de marque (marque de grande distribution ou nationale) ;
				\item qualité.
			\end{itemize}

		\end{itemize}
		\n
\begin{comment}		
		\underline{Avantages de la méthode} :
		
		\begin{itemize}
			\item ??
			\item le consommateur n'est pas obligé de se justifier, résultats plus cohérents ;
			\item ??
			\item intérêt stratégique : on peut identifier la concurrence immédiate (les marques les plus proches) ;
			\item si on n'est aucun des acteurs, on peut identifier un produit voulant rentrer sur le marché (ex : marque artisanale et qualité moindre dans le schéma précédent ; c'est une niche vide).
		\end{itemize}
		
		\underline{Désavantages} : 
		\begin{itemize}
			\item ???
			\item parfois impossible d'interpréter des écarts
		\end{itemize}
\end{comment}
		Conclusions (cfr annexe aussi) :
		
		\begin{itemize}
			\item[-] difficulté d'interprétation
			\item[-] sensibilité à la subjectivité de l'analyste
			\item[+] moindre dépendance des hypothèses de départ
			\item[-] lourdeur du questionnaire
			\item[+] meilleure spontanéité des répondants
		\end{itemize}
		
		\subsection{L'analyse factorielle}
		
		Pour n marques et p attributs, on demande de noter chaque marque sur les attributs. 
		
	%	Ex produit de maquillage p288.
		
		
		\begin{itemize}
			\item[+] facile d'interprétation ;
			\item[+] moindre dépendance de l'analyste ;
			\item[-] sensibilité aux hypothèses de départ
			\item[-] lourdeur du questionnaire
			\item[-] effet de rationalisation à craindre ;
		\end{itemize}
		
		\subsection{Natyr}
		
		Superposition des deux cartes.
		
		
	
	\section{Le ciblage}
	
	
		\subsection{Choisir combien de segments viser}
		
		
			\subsubsection{Dilemme standardisation-adaptation}
			
			\paragraph{Standardisation}
			Logique de production ; accroître la productivité par une standardisation des produits et répondre aux besoins de prix faible. A l'extrême : contre-segmentation. \\
			
			\paragraph{Adaptation}
			Logique de marketing ; on s'efforce de diversifier les besoins, développement sur mesure.
			
			
			\begin{itemize}
				\item \underline{Ciblage indifférencié} : traiter le marché comme un tout et mettre l'accent sur ce qu'il  a de commun dans les besoins (standardisation). \\
				
				Condition d'application favorable : demande assez homogène 
				
				\begin{itemize}
					\item[+] économie d'échelle
					\item[-] mauvaise réponse à une éventuelle diversité
				\end{itemize}
				
				
				\item \underline{Ciblage différencié (complet ou partie)} : s'adresser à la totalité (ou une partie) du marché avec des solutions adaptées à chaque segment (adaptation).   \\
				
				Condition d'application favorable : demande hétérogène avec choix de couvrir (presque tout) le marché
				
				\begin{itemize}
					\item[+] fortes parts de marché
					\item[-] coûts élevés
				\end{itemize}
				
				
				\item \underline{Ciblage concentré} : se spécialiser sur un ou deux segments. \\
				
				Condition d'application favorable : spécialisation et demande hétégorène sans choix de couverture large
				
				\begin{itemize}
					\item[+] convient aux ressources limitées
					\item[-] fragilité importante
				\end{itemize}
				
				
				
				\item \underline{Sur-mesure}
			\end{itemize}

			
			\dessin{21}
			
			
		\subsection{Choisir quels segments viser}
		
		Il faut tenir compte de la concurrence et de la demande d'une partie du marché.
		
		Pièges de la niche vide :
		
		\begin{itemize}
			\item y a-t-il assez d'acheteurs ?
			\item y a-t-il un potentiel de profit suffisant ?
			\item est-ce techniquement faisable ;
			\item un positionnement sur cette niche serait-il crédible et cohérent ?
		\end{itemize}
		
		Première question à se poser : pourquoi les concurrents n'y sont-ils pas déjà ?
		
	
		7 étapes pour choisir une place différente (compétitive) et appréciée (attractive) :
		
		\begin{enumerate}		
			\item repérer les attributs importants du marché ;
			\item segmenter par avantages recherchés ;
			\item trouver un "proxi" (descriptif, ...) ;
			\item dessiner la carte de segmentation ;
			\item dessiner la carte perceptuelle ;
			\item superposer les deux cartes ;
			\item sur la double carte (carte de positionnement) choisir, si possible, un positionnement attractif et compétitif.
		\end{enumerate}
		
	\section{Le positionnement}
	
	%Il faut regarder la cohérence.
	
	Une fois les segments-cibles choisis, il faut décider du positionnement à adopter. Il s'agit de l'acte de conception d'une marque et de son image dans le but de lui donner, dans l'esprit de l'acheteur, une place appréciée et différente de celle occupée par la concurrence. \\
	
	Ce n'est pas ce que l'on fait au produit, mais ce qu'on fait à l'image de ce produit dans l'esprit des consommateurs.
	
		\subsection{L'image de marque}
		
		Ensemble des représentations mentales (affectives et cognitives) que des individus associent à une entreprise ou à une marque. On représente souvent ce jugement mental sur des cartes perceptuelles, qui placent les produits proches ou lointains les uns des autres selon leur concurrence perçue. \\
		
		Intérêt des cartes :
		
		\begin{itemize}
			\item identifier les dimensions privilégiées par les consommateurs ;
			\item repérer des sous-groupes de produits perçus comme proches (donc concurrents)
			\item détecter les niches vides où il serait intéressant de développer un nouveau produit
			\item confronter le positionnement perçu par les consommateurs à celui voulu par l'entreprise, afin d'éventuellement effectuer des corrections.
		\end{itemize}
		
		Une image de marque est perceptuelle : elle ne correspond pas nécessairement à une réalité objective. L'image peut être très différente du positionnement que l'entreprise souhaite.
		
		\subsection{Analyse du positionnement}
		
		Cela consiste à décrire la diversité de l'offre, c'est-à-dire quels sont les groupes de produits perçus sur un marché. Ces groupements sont révélés dans les cartes perceptuelles qui identifient les différents paniers d'attributs recherchés.
		
		Après avoir dégagé les groupes de produits, on voit si la variété de l'offre (différenciation) rencontre la variété de la demande (segmentation) : on tente de trouver les correspondances entre une carte de positionnement et une carte de segmentation. 
		
		Dans certains cas, les décideurs peuvent choisir d'associer un produit à des produits qui ne lui est pas naturel, afin par exemple d'éviter une concurrence trop direct, ou de servir un segment plus attractif.
		
		
		\subsection{Choix d'un positionnement}
		
		Si une entreprise ne choisit pas son positionnement, c'est le marché qui s'en chargera, qu'elle le veuille ou non, bon ou mauvais. C'est de plus un choix stratégie à long terme, qui servira de socle décisionnel : il faudra choisir entre un positionnement imitatif (me-too) ou un positionnement différencié. \\
		
		Un positionnement est imitatif lorsqu'on cherche à coller à un concurrent, en reprenant son positionnement et en se présentant comme un substitut. Cela permet de réduire les investissements et les risques de lancement, puisque le segment est déjà ouvert. Le désavantage est qu'il sera compétitif.
		
		Une positionnement est différencié quand, justement, on évite la concurrence. Il y a différentes manières de se différencier :
		
		\begin{enumerate}
			\item différenciation par attributs supérieurs (autre que le prix et l'image)
			\item différenciation par le prix
			\item différenciation par l'image
		\end{enumerate}
		
		N'importe quelle différenciation n'est pas nécessairement efficace.
			
		
		\subsection{Repositionnement}
		
		Dilemme : garder ou non le même nom ? Oui pour éviter la perte de notoriété, non pour éviter l'effet d'inertie.
	
	
	\section{La couverture internationale}
		Analyse de l'environnement international : 4 solutions différentes (forces globales en faveur de la standardisation/adaptation) : 
		
		\dessinS{61}{.65}
		
		\begin{enumerate}
			\item environnement multi-domestique : dominé par les particularités locales ou les réglementations propres à chaque pays
			\item environnement international placide : les forces globales et locales sont faibles, il n'y a pas de stratégie ou de mode d'organisation dominant
			\item environnement global : les forces qui poussent à la standardisation sont puissantes, et ne sont pas compensées par des forces locales très fortes
			\item l'environnement transnational : fortes pressions en faveur de la standardisations, mais où les forces locales sont également très présentes.
		\end{enumerate}
		
		\subsection{Compromis standardisation/adaptation}
		
		Recherche d'un compromis entre standardisation et adaptation en marketing global ; on vise les similarités transnationales tout en s'adaptant aux différences locales.

		\dessinS{62}{.65}
		
		Selon les attentes du marché et l'importances des différences culturelles, on choisit une politique de produit :
		\begin{enumerate}
			\item un produit existant : produit physiquement identique dans chaque pays (excepté l'étiquetage et la langue utilisée)
			\item un produit adapté : basiquement le même, mais modifications apportées au niveau de la langue, des couleurs, du conditionnements, etc pour coller aux réglementations, aux exigences climatiques, de goûts, etc.
			\item un produit nouveau : spécialement conçu pour les besoins de chaque pays.
		\end{enumerate}
		