
\chapter{L'analyse des besoins par la segmentation}

	La première étape d'une élaboration de stratégie marketing est de définir le marché de référence. Pour cela, on va procéder à la segmentation du marché, composée de deux étapes :
	
	\begin{itemize}
		\item une macro-segmentation : identification des produits-marchés ;
		\item une micro-segmentation : identification des segments à l'intérieur de chacun des marchés retenus.
	\end{itemize}
		
	C'est sur base de ce découpage qu'on pourra évaluer l'attractivité, mesurer sa compétitivité et décider d'un ciblage et d'un positionnement.
	
	\section{La macro-segmentation}
	
		\subsection{Conceptualisation du marché de référence}
		
		Trois questions de conceptualisation pour définir le marché de référence d'une entreprise :
	
		\begin{itemize}
			\item quels sont les solutions ou les fonctions à pourvoir ?
			\item quels sont les groupes de clients potentiels intéressés par ces besoins ?
			\item quelles sont les solutions technologiques existantes ou les métiers susceptibles de rencontrer ces besoins ?
		\end{itemize}
	
		\dessin{10}
	
		%\dessin{11}	
	
	
	
		\subsection{Utilité d'une macro-segmentation}
		
		La macro-segmentation sera utilisée si on peut avoir
		\begin{enumerate}
			\item une délimitation précise de son marché, cela envoie un message clair aux personnes appartenant au contexte d'analyse, peut faire passer des menaces futures ;
			\item un repérage de la concurrence ;
			\item un diagnostic de dispersion (ou concentration/synergie) : on sait si l'entreprise se concentre sur certaines zones ou si les activités sont éparses ;
			\item une génération d'idées de nouveaux produits-marchés ;
			\item un choix d'un stratégie de couverture :
			
			\begin{itemize}
				\item concentration sur un produit-marché (recherche d'une part élevée de marché dans un créneau défini) ;
				\item spécialiste-produit, spécialiser sur une fonction ; satisfaction d'un besoin en couvrant tous les groupes de clients concernés (ex : entreprises fabriquant des composants);
				\item spécialiste-client:  concentrer sur un groupe de clients (ex : hôpitaux) ;
				\item spécialisation sélective, introduire plusieurs produits dans des marchés sans liens entre eux (stratégie opportuniste, diversification);
				\item couverture complète, assortiment complet pour rencontrer les besoins de tous les clients ; la totalité des marchés est couvert.
			\end{itemize}			
			
			\item une préparation d'une micro-segmentation.
		\end{enumerate}
		
		
	
	\section{La micro-segmentation}
	
	%Segmentation : découper marché selon les besoins. La différenciation découpe le marché selon la variété de l'offre. 

	L'objectif de la (micro-)segmentation est d'analyser la diversité des besoins de différents groupes de clients de marchés. On supposent que les clients recherchent la même fonctionnalité de base, mais peuvent avoir des attentes ou des préférences variées au niveau des fonctionnalités périphériques ; c'est le principe de la géométrie variable. \\
		
	Exemple : le besoin de base des entrepreneurs est de trouver des moyens pour investir et se développer. Le besoin de base des gestionnaires est de trouver des moyens d'épargne et de se maintenir. \\
		
	On tente donc d'identifier des sous-groupes de clients qui recherchent les mêmes fonctionnalités du panier d'attributs. \\
		
	\textbf{La segmentation n'est pas la différenciation} ; la différenciation décrit la diversité de l'offre, alors que la segmentation représente la diversité de la demande.

	%Principe de la majorité fallacieuse (niche inexploitée) : Il vaut mieux être leader sur un petit segment qu'un "petit joueur" sur un gros segment.
		
		\subsection{Etapes de la démarche de positionnement}
		
		Il y a 3 grandes étapes :
			
		\begin{enumerate}
			\item segmentation : découper le marché en segments homogènes du point de vue des avantages recherchés, mais différentes les uns des autres (condition d'hétérogénéité) ;
			\item ciblage : sélectionner un ou plusieurs segments cibles ;
			\item positionnement : se positionner dans chacun des segments cibles retenus en développant un programme opérationnel (les 4P) ciblé (en tenant compte des positions détenues par la concurrence).
		\end{enumerate}
			
					
			%\subsubsection{la segmentation (décrire)}
			
			%\begin{enumerate}
			%	\item identifier les segments de marché
		%		\item évaluer les segments identifiés
		%	\end{enumerate}
			
		
		Pour découper le marché en segments homogènes, 5 manières :
			
		\begin{itemize}
			\item segmentation par avantages recherchés ;
			\item segmentation descriptive ;
			\item segmentation comportementale ;
			\item segmentation socioculturelle ;
			\item segmentation par occasion d'achat.
		\end{itemize}
			
			%\begin{enumerate}
				%\item segmentation par avantages recherchés, basée sur les attentes du client \\
			\subsubsection{Segmentation par avantages recherchés}				
			Basée sur les attentes du client. \\
				
			\underline{Critère de construction} : importances accordées	aux différents attributs (fonctions, valeurs) d'un produit. \\
				
			\underline{Utilité} : 
			\begin{itemize}
				\item meilleure cohérence avec les choix parfois opportunistes...
				\item très prédictive des comportement
			\end{itemize}
			\n
			\underline{Limites} : 
			\begin{itemize}
				\item identification difficile des attributs
				\item faible connaissance du profil
			\end{itemize}			
				
				
				
				%\item segmentation descriptive, basée sur le profil socio-démographique \\
				
			\subsubsection{Segmentation descriptive}
				
			Basée sur le profil socio-démographique. \\				
				
			\underline{Critère de construction} : âge, sexe, revenus, localisation, profession, niveau d'étude \\
				
			\underline{Utilité} : 
			\begin{itemize}
				\item identification facile
				\item statistiques préexistantes
			\end{itemize}
				\n
			\underline{Limites}
			\begin{itemize}
				\item a priori (pas explicatif)
				\item de moins en moins prédictive des comportements (car les choix deviennent opportunistes)
			\end{itemize}
			\n	
				%[dessin p 198]
				
				
			Exemple de nouveaux segments socio-démographique :
				
			\begin{enumerate}
				\item le segment des seniors (3 et 4ème âge)
				\item le segment des ménages d'une personne (célibataire, divorcés, veufs)
				\item le segment des ménages à double revenu (pouvoir d'achat élevé, peu de temps libre)
				\item le segment des femmes au travail (recherche de gain de temps)
			\end{enumerate}
				
				
%				\item La segmentation comportementale \\
					
			\subsubsection{Segmentation comportementale}			
				
			\underline{Critères de construction} : 
				
			\begin{itemize}
				\item taux d'utilisation du produit
				\item vitesse de réaction à l'innovation 
				\item statut d'utilisateur (non utilisateur, utilisateur potentiel, 1er utilisateur, utilisateur réguliers ou irréguliers)
				\item statut de fidélité (inconditionnels, non exclusifs, mercenaires, terroristes)
			\end{itemize}
				\n
			\underline{Utilité} : tient compte des réponses différenciées (CRM) \\
				
			\underline{Limites} : a postériori, donc peu prédictive
				
						
				%\item la segmentation socioculturelle, par style de vie
				
			\subsubsection{Segmentation socioculturelle}
				
			Par styles de vie \\				
				
			\underline{Critères} : 
				
			\begin{itemize}
				\item valeurs : croyances fermes et durables
				\item activités, intérêts, opinions
				\item produits achetés
			\end{itemize}
				\n
			\underline{Utilité} : 
				
			\begin{itemize}
				\item bon suivi de l'évolution des sensibilités
				\item meilleur prédiction
			\end{itemize}
				\n
			\underline{Limites} :
				
			\begin{itemize}
				\item typologies parfois divergentes
				\item manque de modèle explicatif valide
			\end{itemize}
				
				
				
				%\item la segmentation par occasions d'achat

			\subsubsection{Segmentation par occasions d'achat}				
				
			\underline{Critères de construction} :
			\begin{itemize}
				\item moment
				\item lieu
				\item contexte
			\end{itemize}
				\n
			\underline{Utilité} :
				
			\begin{itemize}
				\item bonne proximité des avantages recherchés
				\item prise en compte des comportements opportunistes
				\item possibilités de prix flexibles (lissages des pics)
			\end{itemize}
				\n
			\underline{Limites} :
				
			\begin{itemize}
				\item exposé à la versatilité des comportements
				\item l'utilisation des prix flexibles sous conditions (yield) % (consentement  payer. Possibilité de segmentation. Pas possibilité d'arbitrage. Faible coût de mise en place. Pas de perception d'injustice). % Qu'ai-je voulu dire par là ? call(3615 VOYANT).
			\end{itemize}
				
			%\end{enumerate}

			%\subsubsection{le ciblage (décider)}
			
			%What
			%\subsubsection{le positionnement (développer)}
			
		%	The
		
	%	Fuck??
		
		
		
		
	\section{Les tribus}
	
	Tribu : regroupement spontané d'individus autour d'un intérêt commun. \\
	
		\subsection{Spécificités}
	
		\begin{itemize}
			\item émergence spontanée
			\item regroupement réel : sentiment d'appartenance
			\item affiliation plurielle et éphémère : entrée et sortie en fonction des affinités
		\end{itemize}
	
	
		\subsection{Utilité}
		
		Une segmentation tribale peut être utile :
		
		\begin{itemize}	
			\item différenciation basée sur le lien véhiculé par le produit
			\item fidélisation par appartenance à une communauté
			\item image basée sur la culture tribale
			\item communication bouche-à-oreille (buzz) avec fort pouvoir de conviction par les pairs
		\end{itemize}
	
	\section{Les conditions d'efficacité d'une segmentation}
	
	Une segmentation est efficace et utile si les segments rencontrent 4 groupes de condition :
	
	\begin{enumerate}
		\item Réponse différenciée, ce qui signifie
		\begin{itemize}
			\item hétérogénéité entre segments et
			\item homogénéité à l'intérieur des segments
		\end{itemize}
		
		Si cette condition n'est pas respectée, risque de cannibalisme ;
		
		
		\item Taille suffisante : si la condition n'est pas respectée, non rentable.;
		\item Mesurabilité : si la condition n'est pas respectée, abstraction irréaliste ;
		\item Accessibilité :
		\begin{itemize}
			\item auto-sélection ou
			\item couverture contrôlée
		\end{itemize}
		
		Si la condition n'est pas respectée, dilution des efforts ;
		
		\item (Stabilité dans le temps : les gens ne changent pas de tribus. Critère non considéré.)
	\end{enumerate}
	

	
	\dessin{12}
		