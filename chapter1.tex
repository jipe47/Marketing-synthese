\chapter{La compréhension des besoins du client}


	\section{Notion d'un besoin}

	Le but du marketing est de satisfaire les besoins du client et la première démarche est d'identifier le profil du client dans un marché. \\




\begin{comment}
	Un \textbf{besoin} est un état de manque de quelque chose de fondamental, insaturable, générique, propre à l'activité humaine et couvre :

	\begin{itemize}
		\item l'inné (nature, organisme) ;
		\item l'acquis socioculturel  (dépend de l'expérience, de la société, de l'environnement, ...).
	\end{itemize}
	
\end{comment}


		\subsection{Les types de besoins}

		Un besoin est une exigence de la nature ou de la vie sociale. Il y a deux sortes de besoin :

		\begin{itemize}
			\item les besoins génériques, insaturables, inhérents à la nature ou à l'organisme ;
			\item les besoins acquis, culturels et sociaux (dépendent de l'expérience, de la société et de l'environnement).
		\end{itemize}
		\n
		Le marketing voit les besoins génériques comme des problèmes auxquels sont confrontés les clients qui cherchent des solutions. On va alors distinguer les besoins dérivés, qui sont des réponses technologiques particulières aux besoins génériques et objets du désir (ex : automobile est le besoin dérivé par rapport au besoin générique de transport individuel autonome). Les besoins dérivés sont saturables.

		\subsection{Les désirs}
		
		Un \textbf{désir} est une solution technologique particulière à un besoin générique, un moyen concret de satisfaire un besoin. C'est saturable et le marketing donne ces moyens.

		Le marketing peut créer un désir, mais pas un besoin ; il peut créer une frustration ou exercer certains désirs, ainsi que mettre à jour un besoin générique enfouis. 

		Le client n'achète pas un produit mais un service apporté par le produit. Le résultat est un besoin stable dans le temps ; les solutions technologiques changent rapidement. Il faut donc faire une approche selon le client.

		Ex : Les Tetrapacks ne sont pas des boîtes en carton mais des moyens de conservation de produits alimentaires liquides.

		\subsection{Les besoins selon Maslow}

		Pyramide de Maslow :

		\begin{enumerate}
			\item besoins d'accomplissement (développement personnel) ;
			\item besoins d'estime (dignité, compétence) ;
			\item besoins sociaux (appartenance, reconnaissance) ;
			\item besoins de sécurité (intégrité, protection) ;
			\item besoins physiologiques (faim, soif).
		\end{enumerate}

		\dessin{1}

		
	\section{Notion d'un produit}

	Un produit est un bien ou un service ou une combinaison des deux ; ensemble de valeurs, de paniers d'attributs (ou de fonctionnalités ou de services rendus).


	Un produit a 4 types de fonctionnalités, réparties en 2 catégories :

	\begin{enumerate}
		\item les fonctionnalités de base (ex : laver la vaisselle) : valeur fonctionnelle de la classe de produit ;
	
		\item les fonctionnalités périphériques :
	
		\begin{enumerate}
			\item les fonctionnalités nécessaires : modalités de production du service de base (confort, économie, absence de bruit) et ce qui accompagne la fonction de base (emballage, SAV).
			\item les fonctionnalités ajoutées (installation, services, livraison et crédit, garantie, service après-vente) : services offerts en plus par la marque, représente un éléments distinctif important ;
			\item les associations mentales (émotions, personnalités, logo, symboles, pays d'origine).
		\end{enumerate}
	\end{enumerate}
	\n
	Le service de base d'un produit définit le marché de référence d'une entreprise, pour les raisons suivantes : 

	\begin{enumerate}
		\item l'acheteur recherche la fonctionnalité de base procurée par le produit, et pas le produit en tant que tel ;
		\item des produits technologiquement très différents peuvent apporter la même fonction de base ;
		\item les technologies peuvent évoluer rapidement et profondément, alors que les besoins auxquels répond la fonction de base restent stables.
	\end{enumerate}
	\n
	Les fonctionnalités nécessaires sont le terrain de la concurrence et de la compétition.

		\subsection{Multi-dimensionnalité d'un produit}
		
		Un produit est vu comme un panier d'attributs qui satisfont le client, et qui peuvent être très différents selon les marques. Le producteur voit les choses différemment.
		
		\dessinS{57}{0.7}
		
		On a ainsi, pour $j$ clients, $i$ attributs $A_i$ avec une importance $w_{i,j}$ par individu, $k$ marques avec la performance $x_{i, j, k}$ perçue de l'attribut $i$ par l'individu $j$ et $U_{j, k}$ l'utilité totale de l'individu $j$ pour la marque $k$ : 
		
		\dessin{27}
		
		
%\begin{itemize}
%	\item caractéristiques objectives ;
%	\item attributs :
%	\begin{itemize}
%		\item importance ;
%		\item performance.
%	\end{itemize}
%	\item utilités partielles ;
%	\item utilités totales.
%\end{itemize}


		Multi-dimensionnalité des biens :

		\begin{itemize}
			\item caractéristique technique (être) : caractéristique physique descriptive de ce que le produit est ou a ;
			\item service (ou attribut, fonction, avantage) (faire) : ce que la caractéristique fait pour le client ;
			\item valeur (recevoir) : ce que le client reçoit qui correspond à son besoin.
		\end{itemize}

		L'avantage de la valeur et du service est que c'est compréhensible ; les caractéristiques sont là pour convaincre les techniciens.


		\subsection{Types de biens de consommation}

		\begin{enumerate}
			\item biens d'achat courant : biens que le consommateur achète souvent
	
			\begin{enumerate}
				\item de première nécessité: les plus courant (boissons,...) ;
				\item d'achat impulsif : achetés sans préméditation (chips, friandises,...) ;
				\item de dépannage : achetés au moment où le besoin se fait sentir (aspirine, parapluie quand il pleut,...).
			\end{enumerate}
			
			\item biens d'achat réfléchi : biens où le risque est moyen, et pour lesquels le client compare des marques selon des critères ;
			
			\item biens de spécialité : biens dont les caractéristiques sont uniques et pour lesquels le consommateur est prêt à consacrer beaucoup d'efforts (produits de luxe). Généralement pas de comparaisons, car le consommateur sait ce qu'il veut ; la fidélité pour une caractéristique ou une marque est déterminante ;
			
			\item biens non recherchés : biens que le client ne connait pas ou qu'il n'a pas spontanément envie d'acheter. Ex : produits sophistiqués, assurances-vie, encyclopédies,... Importants efforts de vente.
		\end{enumerate}


		\subsection{Les services}
		
		Spécificités des services :

		\begin{enumerate}
			\item intangibles (ne peut être touchés ou vu), immatériels
			\item périssables (pas stockés, capacité de production limitées)
			\item inséparables (les services sont produits et consommés au même moment)
			\item qualité variable (le facteur humain est important)
		\end{enumerate}


	\section{Notion de client}

	\dessin{2}

	Un client peut exercer plusieurs rôles :
	
	\begin{itemize}
		\item si c'est un client consommateur (B2C) :
		\begin{itemize}
			\item Consommateur/utilisateur : utilise le produit ;
			\item Acheteur : obtient le produit ;
			\item Payeur.
		\end{itemize}
		
		\item si c'est un client organisationnel (B2B) :
		
		\begin{itemize}
			\item Consommateur/utilisateur : utilise le produit ;
			\item Acheteur : obtient le produit ;
			\item Payeur ;
			\item Prescripteur : celui qui recommande ou impose le choix ;
			\item Décideur : personne ayant le pouvoir d'engager l'entreprise vis-à-vis d'un fournisseur (souvent les dirigeants de l'entreprise lors d'achats importants) ;
			\item Filtre.
		\end{itemize}
	\end{itemize}
	
	Une grande partie de l'activité commerciale porte sur des transactions entre organisations.
	
		\subsection{Le client organisationnel}

		\subsubsection{Une demande dérivée}

		La demande industrielle est dérivée, c'est-à-dire exprimée par une organisation utilisant les produits achetés dans son propre système de production pour pouvoir répondre à la demande d'autres organisations ou de l'acheteur final. De ce fait, la demande industrielle est \underline{dépendante de la demande aval}.

		\subsubsection{Une demande parfois fluctuante et inélastique}

		La demande industrielle (et notamment la demande d'équipement) est fluctuante et réagit, avec retard, à une faible variation de la demande finale, car la demande d'équipement dépend des capacités de productions des entreprises clientes. Cette dépendance explique les fortes variations dans les demandes de ces biens.

		De plus, la demande industrielle est souvent inélastique au prix, dans la mesure où le produit représente une part peu importante du prix de revient du client.

		\subsubsection{Un client à structure collégiale}

Les décisions d'achat sont souvent prises par un groupe de personnes (centre décisionnel d'achat).

		%\subsubsection{Un client à besoins multiples}
		%[Voir p98-100] non abordé (à confirmer)


		%\subsubsection{Un client menant une conduite résolutoire extensive}
		% non abordé
		\subsubsection{Un produit bien défini, d'importance stratégique}

		Un produit recherché est généralement bien défini par le client qui sait ce qu'il veut. Il y a peu de marge de manoeuvre et le cahier des charges est clairement établie. Les produits achetés entrent dans le système productif du client et ont ainsi une importance stratégique pour lui. \\

Les produits industriels ont souvent un grand nombre d'utilisations possibles, au contraire des biens de consommation qui sont presque toujours à utilisation spécifique.

