
\chapter{L'attractivité des segments}
	
	\section{Concept de demande primaire et de demande à l'entreprise}
		
	
	Demande (cycle de vie) : volume qui serait acheté en réponse à une pression marketing évoluant avec le temps ; demande = $f(P, t)$ \\
	
	Il y a deux sorte de demande : la demande primaire et la demande de l'entreprise.
		
		\subsection{Demande primaire}
		Demande primaire : volume qui serait acheté en réponse à une pression marketing (l'ensemble des acteurs) en un lieu et un temps donné ; demande primaire = $f(P)$ \\
		
		La demande primaire peut être soit 
		\begin{itemize}
			\item expansible : lorsque le niveau de la demande est influencé par des facteurs d'environnement et par l'intensité de la pression marketing globale ; un marché porteur est un marché à taux de pénétration de $\leq$ 20\% , opportunités de croissance en développant le marché) ;
			
			\item non expansible : tout accroissement de la pression marketing globale est sans effet sur le niveau de la demande ; un marché saturé est un marché à taux de pénétration de $\geq$ 80\% , développement possible en attaquant la concurrence.
		\end{itemize}		
					
		\dessin{13}
		
		
		\subsection{Demande de l'entreprise}
		
		Demande à l'entreprise (ou à la marque) : part de la demande primaire détenue par l'entreprise à un moment donné dans un segment ou produit-marché déterminé.
		
		$$\text{demande à l'entreprise} = \text{demande} \times  \text{part de marché}$$
		
		
		\subsection{Déterminants de la demande}
		
		Facteurs hors contrôle (demande) :
		
		\begin{itemize}
			\item comportement des clients ;
			\item circuits de distribution ;
			\item structure de la concurrence ;
			\item environnement macro-marketing (ex : vieillissement de la population).
		\end{itemize}
		\n
		Facteurs sous contrôle (demande à l'entreprise) :
		
		\begin{itemize}
			\item produit ;
			\item distribution ;
			\item prix ;
			\item communication ;
			\item services ;
			\item etc.
		\end{itemize}
		
	\section{Recherche des opportunités de développement de la demande}
	
		\dessin{14}
		
		%En dessous de la ligne : attaquer la concurrence ; au-dessus : développer la demande.
		Lorsqu'on est en dessous de la ligne, il faut attaquer la concurrence.
		
		Lorsqu'on est au-dessus de la ligne, il faut développer la demande.
		
		\subsection{Analyse des écarts entre demande actuelle et marché potentiel}
		
		On peut citer 3 causes pour justifier l'écart entre la demande actuelle et le marché potentiel :
		\begin{enumerate}
			\item déficits d'usage : le produit n'est pas très utilisé. On peut distinguer 3 situations de consommateurs :
			\begin{itemize}
				\item non-utilisateurs
				\item irréguliers
				\item faible quantité
			\end{itemize}
				
			
			\item déficits de distribution :  
			: insuffisance de couverture du marché par le réseau de distribution. 3 situations possibles : 
			
			\begin{itemize}
				\item couverture insuffisante
				\item intensité de distribution insuffisante
				\item exposition insuffisante
			\end{itemize}
	
		
		\item déficits de gamme : inadaptation des produits existants aux différentes situations de consommation ou aux attentes des acheteurs : taille, options, style, couleurs, goûts, formes, design, technologies, niveaux de qualité, \dots % Autre blabla p230
		\end{enumerate}
		
		\subsection{Recherche d'un méta-marché}
		
		Permet d'augmenter la "part du client" en augmentant les ventes croisées. Ex : mariage, divorce, déménagement, \dots  Marché en escalier dans le graphique besoins/solutions technologiques/groupe de clients. Potentiel plus grand \\
		
		\underline{Avantages} :
		
		\begin{itemize}
			\item concept aligné sur le point de vue de l'acheteur
			\item identification d'opportunités de croissance (activités liées)
			\item revenu potentiel supérieur
			\item solution complète en un seul endroit, donc générateur d'exclusivité, de fidélité et de confiance
			\item identification des concurrents indirects et potentiels
			\item communication facile
		\end{itemize}
		
		
	\section{Modèle de cycle de vie d'un produit-marché}
		
	D'un produit-marché ou d'un segment, pas d'une marque. \\
	
	Cycle de vie : demande primaire en fonction du temps
	
	\dessin{15}
	
	Un produit-marché est inévitablement substitué.
	
		\subsection{Réactivité à l'innovation}
		
		\dessin{16}
		
		%\subsection{Modèle du cycle de vie d'un produit-marché}
		% Abordé au cours, mais osef ?
		%[page 234]
		
		
		
		\subsection{Implications stratégiques et opérationnelles du cycle de vie d'un produit-marché}
		
			\subsubsection{Phase d'introduction}		
		
			\underline{Environnement} :
		
			\begin{itemize}
				\item caractéristiques : forte incertitude, haut risque financier
				\item demande : en évolution lente, mais expansible (technologie débutante, distribution réticente)
				\item acheteurs : innovateurs (pionniers) et adopteurs précoces (résistance au changement, faible connaissance (usage/produit))
				\item concurrence : limitée, monopole temporaire
			\end{itemize}
			\n
			\underline{Objectifs stratégiques prioritaires} :
			
			\begin{itemize}
				\item stimuler demande
				\item informer de l'existence du produit et de ses avantages (notoriété)
				\item inciter à l'essai
				\item pénétrer le réseau de distribution réticent
				\item choisir un ciblage indifférencié
			\end{itemize}
			\n
			\underline{Programme marketing} :
			
			\begin{itemize}
				\item produit : de base, gamme limitée
				\item distribution : sélective, voire exclusive
				\item prix : élasticité faible, prix d'écrémage ou de pénétration
				\item communication informative (notoriété)
			\end{itemize}
			
			
			\subsubsection{Croissance}
			
			\underline{Environnement} :
		
			\begin{itemize}
				\item caractéristiques : nombreux nouveaux entrants
				\item demande : expansible et fortement croissante, meilleur disponibilité (expansion distribution) - nouveaux concurrents - diffusion technologique
				\item acheteurs : majorité précoce : meilleure connaissance (bouche à oreille) - désir d'essayer - confiance dans le produit
				\item concurrence : croissance mais pacifique car demande expansible
			\end{itemize}
			\n
			\underline{Objectifs stratégiques prioritaires} :
			
			\begin{itemize}
				\item développer ce marché expansible
				\item maximiser le taux d'occupation
				\item créer une image de marque
				\item chercher un positionnement porteur
				\item renforcer la distribution
				\item choisir un ciblage concentré
			\end{itemize}
		\n
			\underline{Programme marketing} :
			
			\begin{itemize}
				\item produit : extension gamme ou service (caractéristiques ajoutées)
				\item distribution : intensive
				\item prix : élasticité élevée (réduire le prix vers la pénétration si pas déjà fait)
				\item communication : positionnement - image de marque
			\end{itemize}
			
			\subsubsection{Turbulence}
			
			Moment où il y a le plus de faillites ; phase la plus dangereuse. \\
			
			\underline{Environnement} :
		
			\begin{itemize}
				\item caractéristiques : restructurations multiples (secteur et entreprises)
				\item demande : expansible mais croissance en décélération
				\item acheteurs : majorité tardive : comparaison des produits - parfois peu fidèles
				\item concurrence : violente avec départ des plus faibles - me too products
			\end{itemize}
			\n
			\underline{Objectifs stratégiques prioritaires} :
			
			\begin{itemize}
				\item maximiser les pdm dans les cibles
				\item segmenter et positionner
				\item commencer à fidéliser
				\item opter pour un ciblage concentré ou différencié
			\end{itemize}
			\n
			\underline{Programme marketing} :
			
			\begin{itemize}
				\item produit : différenciation guidée par la segmentation
				\item distribution : intensive - couverture maximale
				\item prix : basé sur la valeur perçue de la marque, élasticité élevée ; baisser prix et se rapprocher du prix (SOMETHING) = valeur perçue
				\item communication : publicité pour signaler le positionnement
			\end{itemize}
			
			\subsubsection{Maturité}
			\underline{Environnement} :
		
			\begin{itemize}
				\item caractéristiques : lutte de pdm, surveillance PMR, pression sur les prix
				\item demande : maximale et non expansible : croissance au rythme du secteur - demande de remplacement (bien durables : faire offre de reprise)
				\item acheteurs : tout le marché potentiel (retardataires) - marché hyper segmenté : comparaison des produits - fidélisation
				\item concurrence : stable - oligopole - concentration élevée
			\end{itemize}
			\n
			\underline{Objectifs stratégiques prioritaires} :
			
			\begin{itemize}
				\item maximiser le profit (soutenant pdm)
				\item fidéliser par le marketing relationnel				
				\item chercher des nouvelles niches
				\item lancer des innovations de rupture
				\item opter pour un ciblage différencié
			\end{itemize}
			\n
			\underline{Programme marketing} :
			
			\begin{itemize}
				\item produit : différencier sur attributs nouveaux ou améliorés (grande variété de marques et modèles)
				\item distribution : intensive
				\item prix : élasticité très forte (éviter la guerre des prix)
				\item communication : qualités distinctives revendiquées
			\end{itemize}
			
			
			
			\subsubsection{Déclin}
			
			\underline{Environnement} :
		
			\begin{itemize}
				\item caractéristiques : obsolescence technologique-  modification des préférences
				\item demande : décroissante : diminution demande et concurrence - gamme limitée - stabilisation des prix (parfois hausse)
				\item acheteurs : départ vers d'autres technologies - sinon spécialistes
				\item concurrence : déclinante - disparition des concurrents
			\end{itemize}
			\n
			\underline{Objectifs stratégiques prioritaires} :
			
			\begin{itemize}
				\item réduire les dépenses et récolter (dé-marketing)
				\item survivre par la spécialisation
				\item opter pour un ciblage concentré
			\end{itemize}
			\n
			\underline{Programme marketing} :
			
			\begin{itemize}
				\item produit : spécialisation - élagage (gamme limitée)
				\item distribution sélection
				\item éventuellement augmenter les prix pour compenser le rétrécissement du marché
				\item se limiter à une communication réduire et ciblée sur la niche
			\end{itemize}
			
	\paragraph{Mises en garde}
		
	\begin{enumerate}
		\item les cycles de vie n'ont pas tous ce profil "idéal"
		\item se baser sur le seul taux de croissance n'est pas suffisant
		\item le modèle n'est pas déterministe : des innovations peuvent changer les profils, le déclin n'est pas nécessairement inéluctable
		\item il ne faut pas confondre CVP (cycle de vie produit) et CVM (cycle de vie marque).
	\end{enumerate}
		
		
		\subsection{Cycle de vie et des flux financiers}
		
		\dessin{18}
		
		
		