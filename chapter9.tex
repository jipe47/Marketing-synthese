
\chapter{La marque}
	
	\section{La marque comme panier d'attributs spécifiques}
		
	Ce à quoi fait penser le nom d'une marque.
	
	\dessinS{55}{0.65}
	
	Diagramme de Kano.
	
	\section{Les fonctions de la marque}
	
	Pour le vendeur :
	
	\begin{itemize}
		\item protection : contre les contrefaçons ;
		\item positionnement  : outil de repérage = résumé du panier d'attributs ;
		\item capitalisation : cristallisation des retombées publicitaires
	\end{itemize}
	
	
	Pour le client B2C :
	
	\begin{itemize}
		\item garantie : qualité stable toujours et partout
		\item repérage : identifier rapidement les produits
		\item practicité : gain de temps et d'énergie par rachat à l'identique 
		\item personnalisation :
		\item fonction ludique, hédonisme
	\end{itemize}
	
	Pour le client B2B (vendeur):
	
	\begin{itemize}
		\item fonction de positionnement %que B2C
		\item fonction de protection, de traçabilité : engagement de la responsabilité amont ;
		\item fonction de capitalisaiton, facilitateur de performance : rassurer le CDA (centre décisionnel d'achat) (malgré quelques craintes)
	\end{itemize}
	
	
	\section{Positionnement, identité, image}
	
	\begin{itemize}
		\item Positionnement : volonté de l'entreprise
		\item identité : message envoyé ;
		\item message reçu
	\end{itemize}
	
		\subsection{L'image de la marque}
		
		L'ensemble des représentations mentales, tant affectives que cognitives, qu'un individu ou un groupe d'individus associe à une entreprise ou une marque (= message reçu)
		
		Image perçue $\approx$ image réelle $\approx$ image voulue
		
		\dessin{31}
		
		
		\subsection{Identité de la marque}
		
		Prisme d'identité 
		
		\dessin{32}
		
		\begin{itemize}
			\item Le physique : le lieu des caractéristiques physiques, la base de la marque, le positionnement classique
			\item la personnalité : les traits, le tempérament, le caractère de la façon de parler de la marque ;
			\item la relation : le climat de relation inspiré par la marque (ex:  Porsche individualiste et non familiale) ;
			\item la culture : le système de valeur, la source d'inspiration, l'univers de référence de la marque) ;
			\item le reflet : l'image extérieure que donne le possesseur ;
			\item la mentalisation : comment le consommateur se perçoit quand il consomme un produit
		\end{itemize}
		
		\subsection{Le capital de marque}
	
		\begin{itemize}
			\item capital de goodwill : fond commercial, cote d'estime accumulée
			\item (et/ou) : valeur supplémentaire perçue qui s'ajoute à la valeur fonctionnelle d'un produit lorsque celui-ci est associé à une marque
		\end{itemize}
		
	\section{La construction d'une marque forte}
	
	Composantes de la marque :
	
	\begin{itemize}
		\item un nom (positionnement, loi) ;
		\item un logo (couleur, graphisme, forme) ;
		\item un symbole (un signe, un personnage, un animal) ;
		\item une signature (un slogan, une ritournelle, un son).
	\end{itemize}
	
		\subsection{Un nom}
		
		Efficace au positionnement (des contre-exemples existent) :
		
		\begin{itemize}
			\item descriptif du produit ;
			\item différenciant ;
			\item mémorable ;
			\item signifiant ;
			\item attrayant ;
			\item transférable ;
			\item adaptable ;
			\item juridiquement valide : 
			
			\begin{itemize}
				\item distinctif (ni générique, ni descriptif) ;
				\item disponible (pas de marque antérieure identique/similaire par un produit identique/similaire) ;
				\item licite (pas contraire aux bonnes moeurs, pas trompeuse).
			\end{itemize}
		\end{itemize}
		
		
		\subsection{Un logo}
		
		\subsection{Une signature}
		
		
	\section{Stratégies de marques de fabricant}
		
		\subsection{Personnaliser la marque : marque-produit}
		
		Une marque par produit. Affecter un seul nom  un seul produit et un seul positionnement
			
			\begin{itemize}
				\item[+] signal clair de positionnement
				\item[+] éventuelles crises isolées
				\item[-] investissements publicitaires élevés
				\item[-] branduit ? (marque pas bien soutenue)
			\end{itemize}
			
		\subsection{Insérer dans une famille}
		
		
		Marque-ligne : proposer des produits complémentaires très proches
		\begin{itemize}
			\item[+] force et cohérence d'image
			\item[+] extension moins coûteuse
			\item[-] inertie du positionnement
		\end{itemize}
		
		
		Marque-gamme : promesse commune à un ensemble de produits relativement différents
		
		\begin{itemize}
			\item[+] capital de marque
			\item[-] banalisation du nom				
		\end{itemize}
		
		\subsection{Authentifier la source}
		
		
		Marque-ombrelle : nom de famille unique pour des produits-marchés différents
		
		\begin{itemize}
			\item[+] grande liberté
			\item[-] dilution d'image
		\end{itemize}
		
		Marque-source : nom de famille avec des prénoms individualisés
		
		\begin{itemize}
			\item[+] possibilités de différenciation
			\item[-] extension limitées
		\end{itemize}
		
		
		Marque-caution : signature
		
		\begin{itemize}
			\item[+] garantie de qualité
			\item[-] faible récupération de notoriété
		\end{itemize}
		
		
		\subsection{Label}
		
		Pas une marque, joue le rôle de caution (doit être dissocié de l'objectif commercial), par une entreprise de labelisation.
		
		
	\section{Stratégies de marques des distributeurs}
	
		\subsection{Se défendre sur les prix}
		
		Marque générique (marque "drapeau") : emballage ascétique, seule description du contenu. Positionnement "moins bon, premier prix" en réponse aux hard discounter (il y a parfois confusion entre les deux concepts "premiers prix").
		
		
		\subsection{Copier les marques nationales}
		
			\subsubsection{Contre marque}
			La contremarque : marque propre pour capter la clientèle d'un leader ou détrôner un challenger. Copie souvent claire. L'objectif est de détourner la clientèle.
		
			\subsubsection{Marque d'enseigne}
			
			Ne cache pas sa paternité, moins cher, qualité identique. Objectif : différencier l'enseigne et créer une image au magasin.
		
		\subsection{Se différencier des fabricants, innover}
		
		La marque maison : produit de haut de gamme, à valeur unique. Objectif : fidéliser, capter des clients à l'enseigne.
		

	
	\section{Répliques de fabricants}
	
	Différents rapports qualité/prix des MDD (marques de distributeurs):
	\dessin{34}
		
	Répliques possibles :
		
	\begin{itemize}
		\item développer le marketing d'aspiration (pull) ;
		\item court-circuiter les distributeurs par la vente directe (internet, domicile)
		\item se limiter aux petits réseaux moins puissants
		\item se concentrer sur l'activité de production (logistique) et sous traiter le relationnel au distributeur. Se protéger par la multiplicité des clients-distributeurs
		\item utiliser son assise mondiale pour une segmentation/différenciation plus fine
		\item développer le trade marketing
		\item développer leur propre premier prix
	\end{itemize}	
		
	
	\section{Extensions de marque}
	
	Extension de marque : marque-ligne ou marque-gamme 
	
	Expansion de marques : marque-ombrelle, marque-source, marque-caution
	
	\dessin{35}
	
	\dessin{33}

