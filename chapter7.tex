
		
		
\chapter{Choix d'une stratégie marketing}
	

	\section{Analyse du portefeuille}
	
		\subsection{BCG}
	
		Hypothèses implicites d'analyse : 
		
		\begin{itemize}
			\item Modèle du cycle de vie des produits-marchés : le besoin de liquidité financières est élevé en phase de croissance et diminue en phase de maturité ;
			
			\item loi d'expérience : une part de marché relative élevée implique un avantage-coût sur les concurrents. Inversement, une part de marché relative faible implique un handicap-coût.
			
		\end{itemize}
		
		On a la relation
	
	$$\text{ventes} = \text{demande} \times pdm$$
	
	Les parts de marché sont sous contrôle grâce au marketing, mais la demande est souvent incontrôlable.
	
	Il y a 4 positions possibles selon les parts de vente :
	
	\begin{itemize}
		\item star : $1 = 1 \times 1$
		\item enfants à problèmes : $0 = 1 \times 0$
		\item vache à lait : $0 = 0 \times 1$
		\item poids morts : $0 = 0 \times 0$
	\end{itemize}
		
		
		\dessinS{43}{0.45}	
	
	\begin{itemize}
		\item {\color{red}B+} : grand besoin de liquidité 
		\item {\color{green}B-} : besoins de liquidité moindre
		\item {\color{blue}G+} : génération de liquidité élevée
		\item {\color{orange}G-} : génération de liquidité faible
	\end{itemize}
	
	Pour choisir le pivot/la limite de l'attractivité :
	
	\begin{itemize}
		\item moyenne pondérée du marché
		\item taux PNB
		\item autre
	\end{itemize}
		
	Cycle de vie financier d'un produit
		
		
	%\dessinS{44}{0.45}
	%\dessinS{45}{0.45}	
	\dessinS{44-45}{0.45} % combinaison des deux	
	
	\dessinS{46}{0.45}

	Hypothèses :
	\begin{enumerate}
		\item le CVP en S ou déterministe ?
		\item l'expérience : potentiel d'apprentissage, haute VA ou cas exceptionnels ?
		\item en conséquence :
		
		\begin{itemize}
			\item $\uparrow\rightarrow$ : inéluctables (demande)
			\item $\rightarrow\leftarrow$ : volontaristes (compétition)
		\end{itemize}
	\end{enumerate}
	\n
	\begin{itemize}
		\item {\color{red} déplacements de liquidité}
		\item {\color{green} déplacement d'un nouveau produit}
	\end{itemize}
	\n
	Décisions stratégiques :
	
	\begin{itemize}
		\item star : continuer à soutenir
		\item vache à lait : garder, mais ne pas trop investir (car le marché est en déclin) et utiliser ce qui est généré pour l'avenir
		\item poids morts : quitter le marché ou faire profil bas
		\item enfants à problèmes : appliquer une stratégie de suiveur, gagnant avec une force de frappe nécessaire. Si ça rate, devient un poids mort.
	\end{itemize}
	\n
	
		Critiques du système BCG : hypothèses lourdes et pas nécessairement présentes, et les décisions stratégiques sont basées sur deux chiffres. \\
	
			
		\subsection{Grille multi-dimensionnelle}
	
		\dessinS{48}{.65}
	
		Plusieurs indicateurs d'attractivité ; ex : accessibilité du marché, taux de croissance du marché, durée du cycle de vie, potentiel de rentabilité, dureté de la concurrence, potentiel de différenciation, concentration des clients, \dots \\
		
		Plusieurs indicateurs de compétitivité (il faut se comparer à un autre nécessairement) : part de marché relative, coût direct unitaire, qualité distinctives, savoir-faire technologique, organisation commerciale, image de marque, \dots \\

\begin{comment}
		Conseils pour kiwi :
	
		\begin{itemize}
			\item tous les produits d'un marché (ou plus) sur une période
			\item pivot attractivité
			\item tous les produits d'un même segment ont la même attractivité
			\item CLPD = leader ou challenger	
			\item un seul à gauche : le leader ( = 4)
			\item surface = profit ou CA
			\item une couleur par concurrent
			\item prudence yo-yo et proximité des axes
			\item comparaison avec multi-critères : tous les produits
		\end{itemize}
\end{comment}
	
		\subsection{L'analyse SWOT}
	
		(strengths-weaknesses/opportunities-threats)
		
		\begin{center}
		\begin{tabular}{|c|c|c|}
			\hline  & Forces & Faiblesses \\ 
			\hline Opportunités &  &  \\ 
			\hline Menaces &  &  \\ 
			\hline 
		\end{tabular} 
		\end{center}
	
		Audit interne (SW) : forces/faiblesses de l'entreprise $\Rightarrow$ ressources et compétences (capacités stratégiques détenues par l'entreprise) $\Leftrightarrow$ ce qu'on a \\
		
		Audit externe (OT) : opportunités/menaces liées à l'environnement (politique économique, sociale, techno, écologique, légal = PESITEL) et au marché (demande, offre, concurrence, clients, distribution, fournisseurs = DOCoCLiDiF) $\Rightarrow$ facteurs clés du succès (capacités nécessaires) $\Leftrightarrow$ ce dont on a besoin \\
	
		\dessinS{47}{.6}
	
	
		Erreurs courantes SWOT : confusion de concept, listes incomplètes et absence de décision stratégique finale
	
	
	\section{Stratégies de base}

		\subsection{Triple décision}
		
		\begin{enumerate}
			 \item quelle stratégie de base à adopter ?
			 \item stratégie de développement à adopter ?
			 \item stratégie concurrentielle à adopter ?
		\end{enumerate}		
		
		\dessinS{49}{.65}
		
		\subsection{3 orientations de base}
		\dessinS{50}{.65}
		
\begin{comment}	
		\begin{tabular}{|c|c|c|}
			\hline innovation & infrastructure & relation \\ 
			\hline rôle principal : nouveaux produits & logistique, stockage, production & clients, relation \\ 
			\hline clés du succès : vitesse, entrée précoce & échelles, répartition des coûts fixes & économie d'envergure (scope), part des clients \\ 
			\hline culture : centrée sur les travailleurs, stars créatives & orienté-coûts, standardisation, prévisibilité, efficience & orienté-cilent, importance du service \\ 
			\hline compétition : bataille pour les talents, faibles barrières entrée, petits acteurs & bataille pour la taille, consolidation rapide, grands acteurs & bataille pour l'envergure, consolidation rapide, grands acteurs \\ 
			\hline 
		\end{tabular}	
\end{comment}


	\section{Les stratégies de croissance}
		
		\subsection{Croissance intensive}
		
		On essaie de croître au sein du marché de référence. S'applique quand l'entreprise n'a pas exploité toutes les opportunités offertes par les produits dans les marchés qu'elle couvre.
		
		\dessinS{51}{.65}
		
		On essaie ainsi
		
		\begin{enumerate}
			\item de pénétrer le marché :		
			\begin{itemize}
				\item de développer la demande primaire
				\item augmenter la part de marché
				\item acquérir des marchés
				\item défendre une position de marché
				\item rationaliser le marché (réorganisation)
				\item organiser le marché (actions menées par la profession pour améliorer la rentabilité du marché)
			\end{itemize}
			
			\item la croissance par les marchés :
			
			\begin{itemize}
				\item nouveaux segments
				\item nouveaux circuits de distribution
				\item expansion géographique
			\end{itemize}
			\item la croissance par les produits :
			
			\begin{itemize}
				\item ajout de caractéristiques
				\item extension de la gamme de produits ou de marques
				\item rajeunir une ligne de produits
				\item améliorer la qualité
				\item acquérir une gamme de produits
				\item rationnaliser une gamme de produits
			\end{itemize}
		\end{enumerate}
		
		\subsection{Croissance intégrative}
		
		Croissance justifiée si l'entreprise peut augmenter sa rentabilité en contrôlant des activités d'importance stratégique (ex : source d'approvisionnement, réseau de distribution). Les stratégies d'intégration s'opèrent
		
		\begin{enumerate}
			\item vers l'amont : soucis de stabiliser une source d'approvisionnement (matière première, produits semi-finis, composants ou services), et accès à une technologique nouvelle essentielle à la réussite de l'activité (ex : fabricants d'ordinateurs s'impliquent dans la fabrication de semi-conducteurs)
			\item vers l'aval : volonté d'assurer les débouchés. Pour des biens de consommation, il s'agit de contrôler le circuit de distribution. Dans les marchés industriels, on veille à développer les activités de transformation ou d'incorporation. Egalement volonté de mieux comprendre les besoins des clients, avec des filiales pilotes.
			\item latéralement : renforcer la position concurrentielle, en absorbant ou contrôlant certains concurrents.
		\end{enumerate}
		
		
		\subsection{Croissance par diversification}
		
		Se justifie si plus ou trop peu d'opportunité de croissance ou de rentabilité, ou parce qu'il y a trop de concurrence, ou parce que le marché est en déclin.
		\dessinS{52}{.65}
		
		\begin{itemize}
			\item diversification de placement (déploiement) : utiliser un excédent net de liquidités financières dans une activité ayant une rentabilité supérieure ;
			\item diversification de redéploiement : investir dans une activité à forte croissance pour compenser la faible croissance de l'activité principale ;
			\item diversification de survie (relais) : compenser un handicap concurrentiel insurmontable dans une activité nouvelle où les synergies sont favorables ;
			\item diversification de comportement (extension).
		\end{itemize}
		
		
	\section{Les stratégies concurrentielles}
		
		\subsection{Leader}
		
		Position dominante. Choix entre 4 possibilités :
		
		\begin{enumerate}
			\item développement de la demande primaire
			\item stratégies défensives : innovation et avance technologique ; gamme/distribution intensive ; guerre prix/pub
			\item stratégies offensives : maximiser pdm $\Rightarrow$ rentabilité (industries de volume)
			\item stratégie de recul volontaire : dé-marketing ; diversification ; marketing circulaire (lobbying)
		\end{enumerate}
		
		\subsection{Challenger}
		
		Attaque le leader. Choix possibles :
		
		\begin{enumerate}
			\item attaque frontale : mêmes armes, rapport de force nécessaire triple
			\item attaque latérale : point faible
			\item stratégie d'acquisition
		\end{enumerate}
		
		Dans tous les cas, évaluer la capacité de réaction et de défense (vulnérabilité, provocation, représailles).
		
		\subsection{Suiveur}
		
		Plus pacifique que challenger ; coexister pacifiquement, mais créativement. 4 choses à faire :
		
		\begin{enumerate}
			\item segmenter (compétences distinctives)
			\item utiliser les RandD d'amélioration ou de réduction de coûts
			\item penser petit
			\item valoriser le rôle du dirigeant
		\end{enumerate}
		
		\subsection{Spécialiste}
		
		Petite niche de marché. Rester attentif aux conditions de réussite. Conditions de réussite
		
		\begin{enumerate}
			\item potentiel profit suffisant
			\item potentiel de croissante
			\item peu attractif pour concurrence
			\item compétences distinctives défendables
			\item barrières à l'entrée
		\end{enumerate}
			
	\section{Les stratégies de croissance internationale}
	
	Stratégie non obligatoire si on ne s'étend pas sur l'international. \\
	
	Les objectifs :
	
	\begin{itemize}
		\item se protéger de la concurrence : diversification des positions et surveillance des activités concurrentes
		\item se diversifier géographiquement et commercialement
		\item réduire coûts d'approvisionnement
		\item exploiter un excédent de capacité
		\item élargir le marché potentiel
		\item suivre des clients à l'étranger
		\item prolonger le cycle de vie du produit
		\item diversifier le risque commercial : appui sur des clientèles d'environnements économiques différents et de conjoncture plus favorable.
	\end{itemize}