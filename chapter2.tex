

\chapter{L'analyse du processus de réponse}

	\section{Les niveaux de réponse du marché}

	La réponse du marché est l'activité mentale ou physique des acheteurs potentiels survenue en réaction à un incitant marketing. Cette réponse est structurée, intelligible et s'opère par étapes successives, que l'on peut identifier par l'étude de marché. \\

	L'analyse a pour but de comprendre et de retracer le cheminement suivi par l'acheteur. Le but est de prévoir le comportement et de le corriger. \\

	Les réponses peuvent être regroupées en 3 catégories :

	\begin{enumerate}
		\item réponse cognitive (learn) : on mémorise une marque ;
		\item réponse affective (feel) : début d'évaluation, d'estime ;
		\item réponse comportementale (do) : comportement d'achat.
	\end{enumerate}

	%\dessin{3}
	%\dessin{4}
	\dessin{29}
	
	Généralement : learn $\rightarrow$ feel $\rightarrow$ do. Mais le processus peut varier selon le degré d'implication (faible ou forte) et l'appréhension :
	
	\begin{itemize}
		\item le mode intellectuel : appréhension du réel, s'appuie sur la raison ; mode cognitif ;
		\item le mode affectif : s'appuie sur les émotions, l'intuition ; mode sensoriel.
	\end{itemize}
	
	\n
	
	Modes complémentaires, pas toujours faciles à distinguer.
	
	\dessin{28}
	
	
	\section{La mesure de la réponse cognitive}

	Définition de la perception : processus par lequel un individu sélectionne et interprète l'information à laquelle il est exposé. \\
	
	Le niveau le plus simple de la réponse cognitive est la prise de conscience de l'existence d'un produit ou d'une marque. \\
	
	La notoriété d'une marque est la capacité d'un acheteur potentiel d'identifier une marque d'une manière suffisamment détaillée pour la proposer, la choisir ou l'utiliser. \\
	
	3 types de notoriété :
	
	\begin{itemize}
		\item notoriété spontanée (top of mind), lorsqu'on y pense spontanément ;
		\item notoriété assistée (on propose) ;
		\item notoriété qualifiée (connaissance d'expérience, de réputation, de nom,... ou ignorance).
	\end{itemize}
	\n
	
	Les mesures de la connaissance peuvent s'effectuer sur d'autres caractéristiques (style publicitaire, prix, lieux de vente, etc).
	
	\section{La mesure de la réponse affective}

	L'ensemble disponible est l'ensemble des marques, connues ou inconnues, qui existent dans une catégorie de produits.
	
	L'ensemble évoqué est le sous-ensemble de marques qui entrent en considération lors de l'achat : elles ont une probabilité non nulle d'être achetées. Ce sous-ensemble varie dans le temps et en fonction de la situation de consommation.
	
	\dessin{5}
	
	
	
		\subsection{La déterminance d'un attribut}
	
		Critère de choix déterminants et non redondants pour déterminer l'attitude à adopter envers un produit :
	
		\begin{itemize}
			\item performance, évaluation de la présence d'un attribut dans une marque ($x_{i k}$) ;
			\item importance, poids de l'attribut dans l'évaluation	 ($w_i$ ; $\sum_i w_i = 100\%$) ;
			\item différenciation, présence différente d'un attribut d'une marque à l'autre ;
			\item déterminance : importance et différenciation combinées.
		\end{itemize}	
	
		%\dessin{6}	
	
		%$$\text{déterminance} = \text{scores d'importance} \times \text{score de différencation}$$
		
		%$w'_u = w_i . \text{différencation} . norme$ = wut ?
	
		\dessin{30}
		
		Ainsi, le score de l'importance est la somme des performances pondérées par l'importance :
		
		$$\text{Score importance } = \sum_{l = 1}^i x_{i, j} w_{i, j}$$
		
		La différenciation $\sigma_{i, j}$ est l'écart-type d'un attribut de toutes les marques pour un individu.
		
		$$\sigma_{i, j} = \sqrt{ \sum_{l = 1}^k \frac{x_{l, j}^2}{k} - (\sum_{l = 1}^k \frac{x_{l, j}}{k} )^2   } = E[X^2] - E[X]^2$$
	
		La déterminance $w'_{i, j}$ est donnée par moyenne de $w_{i, j} \sigma_{i, j}$
		
		$$w'_{i, j} = \frac{w_{i, j} \sigma_{i, j}}{\sum_{l = 1}^i w_{l, j} \sigma_{l, j}}$$
		
		Le score basé sur la déterminance est la somme des scores de chaque attribut pondéré par la déterminance :
		
		$$\text{Score déterminance } = \sum_{l = 1}^i x_{l, j} w'_{l, j}$$
		
		Il s'agit d'un modèle compensatoire, car les notes faibles sont compensées par les notes élevées d'autre attributs.
	
		\subsection{Les modèles d'attitude non compensatoires}
		
		On distingue 3 types de modèles non compensatoires :
	
		\begin{itemize}
			\item disjonctif : ne considérer que les marques les meilleures sur certains attributs dominants ; ne retenir que les bons ;	
			\item conjonctif : ne retenir une marque que si elle atteint un minimum acceptable pour chaque attribut ; éliminer les mauvais ;
			\item lexicographique : comparer les marques sur l'attribut le plus important ; en cas d'ex-aequo, passer au deuxième attribut ; les meilleurs.
		\end{itemize}
	
	
		\subsection{La matrice importance/performance}
		
		Matrice dans laquelle on place tous les attributs d'une marque.
		
		\dessin{7}
		
		\underline{Pour passer d'un cadre à l'autre} :
		\begin{itemize}
			\item vraies forces : insister, accompagner
			\item vraies faiblesses : stratégie d'amélioration direct/ponctuelle/relative $\rightarrow$ vraies forces
			\item fausses forces : stratégie d'éducation directe/créative $\rightarrow$ vraies forces
			\item vraies faiblesses : stratégie  d'éducation, risquée $\rightarrow$ fausses faiblesses
		\end{itemize}
	
	\n

	\underline{Stratégies de modification de l'attitude} :
	
	\begin{itemize}
		\item Stratégies pour améliorer les performances ou les perceptions du produit :
		
		\begin{enumerate}
			\item Amélioration directe : modifier le produit ;
			\item Amélioration perceptuelle : modifier les croyances à propos de la marque ;
			\item Amélioration relative : modifier les croyances à propos des marques concurrentes.
		\end{enumerate}	
		
		\item Stratégies d'éducation, pour modifier l'importante ou les attentes d'un attribut :
		
		\begin{enumerate}
			\item Stratégie générique directe : modifier  l'importance d'un attribut performant ;
			\item Stratégie générique créative : attirer l'attention sur des attributs non considérés ;
			\item Stratégie générique risquée : minimiser l'importance d'un attribut peu performant.
		\end{enumerate}
		
		\item Stratégies d'accompagnement
		\item Repositionnement de la marque
	\end{itemize}
	
	

	
	\section{La mesure de la réponse comportementale}
	
		%\subsection{Analyse des habitudes d'achat}
		%	Pas vu
		
		La mesure la plus directe et la plus simple est donnée par les statistiques de vente du produit ou de la marque, ainsi que par une analyse du marché.
	
		\subsection{Analyse statique de la part de marché}
		
		$$\text{Part de marché d'une marque} = \frac{\text{Ventes de la marque}}{\text{Total des ventes dans le secteur de la marque}}$$
	
	
		Éléments permettant d'analyser la part du marché :
		\begin{itemize}
			\item taux occupation : rapport du nombre de clients de la marque au nombre total de clients du produit ($\frac{N_m}{N_p}$);
			\item taux d'exclusivité : part des achats du produit des clients de la marque réservée à la marque ($\frac{ \frac{Q_{mm}}{N_m} }{ \frac{Q_{pm}}{N_m} }$) ;
			\item taux d'intensité : rapport des quantités moyennes de produits achetés par les clients de la marque aux quantités moyennes de produit par client du produit($\frac{ \frac{Q_{pm}}{N_m} }{ \frac{Q_{pp}}{N_p} }$).
		\end{itemize}
		
		$$\text{Part de marché } = \text{taux d'occupation} \times \text{taux d'exclusivité} \times \text{taux d'intensivité}$$
		
		Soient les deux marques $m$ et $p$ à comparer. On a
		
		\begin{itemize}
			\item $N_m$ le nombre de clients de $m$
			\item $N_p$ le nombre de clients de $p$
			\item $Q_{mm}$ la quantité de $m$ achetée par les clients de $m$
			\item $Q_{pm}$ la quantité de $p$ achetée par les clients de $m$
			\item $Q_{pp}$ la quantité de $p$ achetée par les clients de $p$
		\end{itemize}
		
		
		$$\text{Part de marché } = \frac{Q_{mm}}{Q_{pp}} = \frac{N_m}{N_p} \times \frac{ \frac{Q_{mm}}{N_m} }{ \frac{Q_{pm}}{N_m} } \times \frac{ \frac{Q_{pm}}{N_m} }{ \frac{Q_{pp}}{N_p} }$$
		
	
	\section{La mesure de la réponse post-comportementale}
	
		\subsection{Analyse dynamique de la part de marché}
	
		Taux de fidélité $\alpha$ : pourcentage de clients (pondérés par le volume d'achat)	qui, après avoir acheté un produit d'une marque, continuent à acheter cette marque. \\
	
		Taux d'attraction $\beta$ : pourcentage de clients qui se sont tournés vers la concurrence ou le non achat d'un produit lors d'une période $t$, et qui rachètent la marque à la période suivante $t + 1$.
	
	
		$$pdm(t + 1) = \alpha pdm(t) + \beta (1 - pdm(t))$$
	
		\dessin{8}
	
		La relation satisfaction-fidélité est équivoque ; même si la satisfaction du client explique principalement sa satisfaction, un client satisfait n'est pas nécessairement un client fidèle, et inversement un client fidèle n'est pas nécessairement satisfait. Elle devrait être linéaire, mais ce n'est pas le cas :
		
		\begin{itemize}
			\item dans les marchés non concurrentiels, la satisfaction a peu d'impact sur la fidélité. Dans ces marchés monopolistiques, les clients n'ont pas le choix (clients captifs) ; une privatisation peut changer brutalement les choses ;
			\item dans les marchés concurrentiels, de nombreux substituts existent et les coûts de transfert sont faibles. Il y a une grande différence entre les clients satisfaits et les clients totalement satisfaits, ces derniers étant des clients réellement fidèles.
		\end{itemize}
	
		\dessin{9}
	
		%Stratégies à développer face à l'insatisfaction : un fort taux de plainte permet de s'améliorer. Il faut les favoriser (numéro d'appel, site web, boîte à suggestions/réclamation, membre du personnel dédicacé, jouer le client-mystère, entretien avec les clients perdus, enquête de satisfaction) et mesurer l'insatisfaction pour pouvoir y apporter une réponse, car les ventes aux clients insatisfaits sont plus faibles, il est coûteux de retenir un client insatisfait, l'acquisition de nouveaux clients est très coûteuse, et les ventes à de nouveaux clients sont souvent plus faibles.
		
		Une stratégie pour faire face à l'insatisfaction est de prendre en compte les plaintes et les favoriser (numéro d'appel, site web, boîte à suggestions/réclamation, membre du personnel dédiée, jouer le client-mystère, entretiens avec les clients perdus, enquête de satisfaction).
		
		L'idée est de mesurer l'insatisfaction et d'y apporter une réponse, car les ventes aux clients insatisfaits sont plus faibles ; il est coûteux de retenir un client insatisfait, l'acquisition de nouveaux clients est très coûteuse, et les ventes à de nouveaux clients sont souvent plus faibles.
		
		En conclusion :
		
		\begin{enumerate}
			\item il faut identifier le degré de(d') (in)satisfaction des clients ;
			\item une plainte n'est pas un élément négatif en soi, car un client accepte un problème dans la mesure où l'entreprise apporte une solution adaptée ; 
			\item les plaintes sont une source d'informations importante pour mieux connaître les attentes des clients et la qualité perçue des produits.
		\end{enumerate}
	